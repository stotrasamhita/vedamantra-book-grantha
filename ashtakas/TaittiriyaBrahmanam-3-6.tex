\sect{𑌷𑌷𑍍𑌠𑌮𑌃 𑌪𑍍𑌰𑌶𑍍𑌨𑌃}
\setcounter{anuvakam}{0}
\dnsub{𑌤𑍈𑌤𑍍𑌤𑌿𑌰𑍀𑌯𑌬𑍍𑌰𑌾𑌹𑍍𑌮𑌣𑍇 𑌤𑍃𑌤𑍀𑌯𑌾𑌷𑍍𑌟𑌕𑍇 𑌷𑌷𑍍𑌠𑌃 𑌪𑍍𑌰𑌪𑌾𑌠𑌕𑌃}

%3.6.1.1
\-\ul{𑌅}\-𑌞𑍍𑌜\-\ul{𑌨𑍍𑌤𑌿} 𑌤𑍍𑌵𑌾𑌮᳴\-\ul{𑌧𑍍𑌵}\-𑌰𑍇 𑌦𑍇᳴\-\ul{𑌵}\-𑌯𑌨𑍍𑌤𑌃᳴।
𑌵𑌨᳴𑌸𑍍𑌪\-\ul{𑌤𑍇} 𑌮𑌧𑍁᳴\-\ul{𑌨𑌾} 𑌦𑍈𑌵𑍍𑌯𑍇᳴𑌨।
𑌯\-\ul{𑌦𑍂}\-𑌰𑍍𑌧𑍍𑌵𑌸𑍍𑌤𑌿᳴\-\ul{𑌷𑍍𑌠𑌾}\-𑌦𑍍𑌦𑍍𑌰𑌵𑌿᳴\-\ul{𑌣𑍇}\-𑌹 𑌧᳴𑌤𑍍𑌤𑌾𑌤𑍍।
𑌯\-\ul{𑌦𑍍𑌵𑌾} 𑌕𑍍𑌷𑌯𑍋᳴ \ul{𑌮𑌾}\-𑌤𑍁\-\ul{𑌰}\-𑌸𑍍𑌯𑌾 \ul{𑌉}\-𑌪𑌸𑍍𑌥𑍇॑।
𑌉𑌚𑍍𑌛𑍍𑌰᳴𑌯𑌸𑍍𑌵 𑌵𑌨𑌸𑍍𑌪𑌤𑍇।
𑌵𑌰𑍍𑌷𑍍𑌮᳴𑌨𑍍𑌪𑍃\-\ul{𑌥𑌿}\-𑌵𑍍𑌯𑌾 𑌅𑌧𑌿᳴।
𑌸𑍁𑌮𑌿᳴𑌤𑍀 \ul{𑌮𑍀}\-𑌯𑌮𑌾᳴𑌨𑌃।
𑌵𑌰𑍍𑌚𑍋᳴𑌧𑌾 \ul{𑌯}\-𑌜𑍍𑌞𑌵𑌾᳴𑌹𑌸𑍇।
𑌸𑌮𑌿᳴𑌦𑍍𑌧\-\ul{𑌸𑍍𑌯} 𑌶𑍍𑌰𑌯᳴𑌮𑌾𑌣𑌃 \ul{𑌪𑍁}\-𑌰𑌸𑍍𑌤𑌾॑𑌤𑍍।
𑌬𑍍𑌰𑌹𑍍𑌮᳴ 𑌵\-\ul{𑌨𑍍𑌵𑌾}\-𑌨𑍋 \ul{𑌅}\-𑌜𑌰𑍞᳴ \ul{𑌸𑍁}\-𑌵𑍀𑌰𑌮𑍍॑॥1॥

%3.6.1.2
\-\ul{𑌆}\-𑌰𑍇 \ul{𑌅}\-𑌸𑍍𑌮𑌦𑌮᳴\-\ul{𑌤𑌿𑌂} 𑌬𑌾𑌧᳴𑌮𑌾𑌨𑌃।
𑌉𑌚𑍍𑌛𑍍𑌰᳴𑌯𑌸𑍍𑌵 𑌮\-\ul{𑌹}\-𑌤𑍇 𑌸𑍗𑌭᳴𑌗𑌾𑌯।
\-\ul{𑌊}\-𑌰𑍍𑌧𑍍𑌵 \ul{𑌊}\-𑌷𑍁𑌣᳴ \ul{𑌊}\-𑌤𑌯𑍇॑।
𑌤𑌿𑌷𑍍𑌠𑌾᳴ \ul{𑌦𑍇}\-𑌵𑍋 𑌨 𑌸᳴\-\ul{𑌵𑌿}\-𑌤𑌾।
\-\ul{𑌊}\-𑌰𑍍𑌧𑍍𑌵𑍋 𑌵𑌾𑌜᳴\-\ul{𑌸𑍍𑌯} 𑌸𑌨𑌿᳴\-\ul{𑌤𑌾} 𑌯\-\ul{𑌦}\-𑌞𑍍𑌜𑌿𑌭𑌿𑌃᳴।
\-\ul{𑌵𑌾}\-𑌘𑌦𑍍𑌭𑌿᳴\-\ul{𑌰𑍍𑌵𑌿}\-𑌹𑍍𑌵𑌯𑌾᳴𑌮𑌹𑍇।
\-\ul{𑌊}\-𑌰𑍍𑌧𑍍𑌵𑍋 𑌨𑌃᳴ \ul{𑌪𑌾}\-𑌹𑍍𑌯𑍞𑌹᳴\-\ul{𑌸𑍋} 𑌨𑌿 \ul{𑌕𑍇}\-𑌤𑍁𑌨𑌾॑।
𑌵𑌿\-\ul{𑌶𑍍𑌵}\-\-\ul{𑍞} 𑌸\-\ul{𑌮}\-𑌤𑍍𑌰𑌿𑌣᳴𑌨𑍍𑌦𑌹।
\-\ul{𑌕𑍃}\-𑌧𑍀 𑌨᳴ \ul{𑌊}\-𑌰𑍍𑌧𑍍𑌵𑌾𑌂 \ul{𑌚} 𑌰𑌥𑌾᳴𑌯 \ul{𑌜𑍀}\-𑌵𑌸𑍇॑।
\-\ul{𑌵𑌿}\-𑌦𑌾 \ul{𑌦𑍇}\-𑌵𑍇𑌷𑍁᳴ \ul{𑌨𑍋} 𑌦𑍁𑌵𑌃᳴॥2॥

%3.6.1.3
\-\ul{𑌜𑌾}\-𑌤𑍋 𑌜𑌾᳴𑌯𑌤𑍇 𑌸𑍁𑌦𑌿\-\ul{𑌨}\-𑌤𑍍𑌵𑍇 𑌅𑌹𑍍𑌨𑌾॑𑌮𑍍।
𑌸\-\ul{𑌮}\-𑌰𑍍𑌯 𑌆 \ul{𑌵𑌿}\-𑌦\-\ul{𑌥𑍇} 𑌵𑌰𑍍𑌧᳴𑌮𑌾𑌨𑌃।
\-\ul{𑌪𑍁}\-𑌨\-\ul{𑌨𑍍𑌤𑌿} 𑌧𑍀𑌰𑌾᳴ \ul{𑌅}\-𑌪𑌸𑍋᳴ 𑌮\-\ul{𑌨𑍀}\-𑌷𑌾।
\-\ul{𑌦𑍇}\-\-\ul{𑌵}\-𑌯𑌾 𑌵𑌿\-\ul{𑌪𑍍𑌰} 𑌉𑌦𑌿᳴𑌯\-\ul{𑌰𑍍𑌤𑌿} 𑌵𑌾𑌚𑌮𑍍॑।
𑌯𑍁𑌵𑌾᳴ \ul{𑌸𑍁}\-𑌵𑌾\-\ul{𑌸𑌾𑌃} 𑌪𑌰𑌿᳴𑌵𑍀\-\ul{𑌤} 𑌆𑌗𑌾॑𑌤𑍍।
𑌸 \ul{𑌉} 𑌶𑍍𑌰𑍇𑌯𑌾॑𑌨𑍍𑌭𑌵\-\ul{𑌤𑌿} 𑌜𑌾𑌯᳴𑌮𑌾𑌨𑌃।
𑌤𑌂 𑌧𑍀𑌰𑌾᳴𑌸𑌃 \ul{𑌕}\-𑌵\-\ul{𑌯} 𑌉𑌨𑍍𑌨᳴𑌯𑌨𑍍𑌤𑌿।
\-\ul{𑌸𑍍𑌵𑌾}\-𑌧𑌿\-\ul{𑌯𑍋} 𑌮𑌨᳴𑌸𑌾 𑌦𑍇\-\ul{𑌵}\-𑌯𑌨𑍍𑌤𑌃᳴।
\-\ul{𑌪𑍃}\-\-\ul{𑌥𑍁}\-𑌪𑌾\-\ul{𑌜𑌾} 𑌅𑌮᳴𑌰𑍍𑌤𑍍𑌯𑌃।
\-\ul{𑌘𑍃}\-𑌤𑌨𑌿᳴\-\ul{𑌰𑍍𑌣𑌿}\-𑌖𑍍𑌸𑍍𑌵𑌾᳴𑌹𑍁𑌤𑌃।
\-\ul{𑌅}\-𑌗𑍍𑌨𑌿\-\ul{𑌰𑍍𑌯}\-𑌜𑍍𑌞𑌸𑍍𑌯᳴ 𑌹\-\ul{𑌵𑍍𑌯}\-𑌵𑌾𑌟𑍍।
𑌤𑍞 \ul{𑌸}\-𑌬𑌾𑌧𑍋᳴ \ul{𑌯}\-𑌤𑌃 𑌸𑍍𑌰𑍁᳴𑌚𑌃।
\-\ul{𑌇}\-𑌤𑍍𑌥𑌾 \ul{𑌧𑌿}\-𑌯𑌾 \ul{𑌯}\-𑌜𑍍𑌞𑌵᳴𑌨𑍍𑌤𑌃।
𑌆𑌚᳴𑌕𑍍𑌰𑍁\-\ul{𑌰}\-𑌗𑍍𑌨𑌿\-\ul{𑌮𑍂}\-𑌤𑌯𑍇॑।
𑌤𑍍𑌵𑌂 𑌵𑌰𑍁᳴𑌣 \ul{𑌉}\-𑌤 \ul{𑌮𑌿}\-𑌤𑍍𑌰𑍋 𑌅᳴𑌗𑍍𑌨𑍇।
𑌤𑍍𑌵𑌾𑌂 𑌵᳴𑌰𑍍𑌧𑌨𑍍𑌤𑌿 \ul{𑌮}\-𑌤𑌿\-\ul{𑌭𑌿}\-𑌰𑍍𑌵𑌸𑌿᳴𑌷𑍍𑌠𑌾𑌃।
𑌤𑍍𑌵𑍇 𑌵𑌸𑍁᳴ 𑌸𑍁𑌷\-\ul{𑌣}\-𑌨𑌾𑌨𑌿᳴ 𑌸𑌨𑍍𑌤𑍁।
\-\ul{𑌯𑍂}\-𑌯𑌂 𑌪𑌾᳴𑌤 \ul{𑌸𑍍𑌵}\-𑌸𑍍𑌤𑌿\-\ul{𑌭𑌿𑌃} 𑌸𑌦𑌾᳴ 𑌨𑌃॥3॥\anuvakamend[\-\ul{𑌸𑍁}\-𑌵𑍀\-\ul{𑌰𑌂} 𑌦𑍁\-\ul{𑌵𑌃} 𑌸𑍍𑌵𑌾᳴𑌹𑍁\-\ul{𑌤𑍋}\-\-𑌽𑌷𑍍𑌟𑍗 𑌚᳴]

%3.6.2.1
𑌹𑍋𑌤𑌾᳴ 𑌯𑌕𑍍𑌷\-\ul{𑌦}\-𑌗𑍍𑌨𑌿𑍞 \ul{𑌸}\-𑌮𑌿𑌧𑌾᳴ 𑌸𑍁\-\ul{𑌷}\-𑌮𑌿\-\ul{𑌧𑌾} 𑌸𑌮𑌿᳴\-\ul{𑌦𑍍𑌧𑌂} 𑌨𑌾𑌭𑌾᳴ 𑌪𑍃\-\ul{𑌥𑌿}\-𑌵𑍍𑌯𑌾𑌃 𑌸᳴\-\ul{𑌙𑍍𑌗}\-𑌥𑍇 \ul{𑌵𑌾}\-𑌮𑌸𑍍𑌯᳴।
𑌵𑌰𑍍𑌷𑍍𑌮᳴\-\ul{𑌨𑍍𑌦𑌿}\-𑌵 \ul{𑌇}\-𑌡\-\ul{𑌸𑍍𑌪}\-𑌦𑍇 𑌵𑍇𑌤𑍍𑌵𑌾𑌽𑌽𑌜𑍍𑌯᳴\-\ul{𑌸𑍍𑌯} 𑌹𑍋\-\ul{𑌤}\-𑌰𑍍𑌯𑌜᳴।
𑌹𑍋𑌤𑌾᳴ 𑌯\-\ul{𑌕𑍍𑌷}\-𑌤𑍍𑌤\-\ul{𑌨𑍂}\-𑌨𑌪𑌾᳴\-\ul{𑌤}\-𑌮𑌦𑌿᳴\-\ul{𑌤𑍇}\-𑌰𑍍𑌗\-\ul{𑌰𑍍𑌭𑌂} 𑌭𑍁𑌵᳴𑌨𑌸𑍍𑌯 \ul{𑌗𑍋}\-𑌪𑌾𑌮𑍍।
𑌮\-\ul{𑌧𑍍𑌵𑌾}\-𑌦𑍍𑌯 \ul{𑌦𑍇}\-𑌵𑍋 \ul{𑌦𑍇}\-𑌵𑍇𑌭𑍍𑌯𑍋᳴ 𑌦𑍇\-\ul{𑌵}\-𑌯𑌾𑌨𑌾॑\-\ul{𑌨𑍍𑌪}\-𑌥𑍋 𑌅᳴𑌨\-\ul{𑌕𑍍𑌤𑍁} 𑌵𑍇𑌤𑍍𑌵𑌾𑌽𑌽𑌜𑍍𑌯᳴\-\ul{𑌸𑍍𑌯} 𑌹𑍋\-\ul{𑌤}\-𑌰𑍍𑌯𑌜᳴।
𑌹𑍋𑌤𑌾᳴ 𑌯\-\ul{𑌕𑍍𑌷}\-𑌨𑍍𑌨\-\ul{𑌰𑌾}\-𑌶𑍞𑌸𑌂᳴ 𑌨𑍃\-\ul{𑌶}\-𑌸𑍍𑌤𑍍𑌰𑌂 𑌨𑍄𑍟𑌃 𑌪𑍍𑌰᳴𑌣𑍇𑌤𑍍𑌰𑌮𑍍।
𑌗𑍋𑌭𑌿᳴\-\ul{𑌰𑍍𑌵}\-𑌪𑌾\-\ul{𑌵𑌾}\-𑌨𑍍𑌥𑍍𑌸𑍍𑌯𑌾\-\ul{𑌦𑍍𑌵𑍀}\-𑌰𑍈𑌃 𑌶𑌕𑍍𑌤𑍀᳴\-\ul{𑌵𑌾}\-𑌨𑍍𑌰𑌥𑍈॑: 𑌪𑍍𑌰𑌥\-\ul{𑌮}\-𑌯𑌾 \ul{𑌵𑌾} 𑌹𑌿𑌰᳴𑌣𑍍𑌯𑍈\-\ul{𑌶𑍍𑌚}\-𑌨𑍍𑌦𑍍𑌰𑍀 𑌵𑍇𑌤𑍍𑌵𑌾𑌽𑌽𑌜𑍍𑌯᳴\-\ul{𑌸𑍍𑌯} 𑌹𑍋\-\ul{𑌤}\-𑌰𑍍𑌯𑌜᳴।
𑌹𑍋𑌤𑌾᳴ 𑌯𑌕𑍍𑌷\-\ul{𑌦}\-𑌗𑍍𑌨𑌿\-\ul{𑌮𑌿}\-𑌡 𑌈᳴\-\ul{𑌡𑌿}\-𑌤𑍋 \ul{𑌦𑍇}\-𑌵𑍋 \ul{𑌦𑍇}\-𑌵𑌾𑍞 𑌆𑌵᳴𑌕𑍍𑌷\-\ul{𑌦𑍍𑌦𑍂}\-𑌤𑍋 𑌹᳴\-\ul{𑌵𑍍𑌯}\-𑌵𑌾𑌡𑌮𑍂᳴𑌰𑌃।
𑌉\-\ul{𑌪𑍇}\-𑌮𑌂 \ul{𑌯}\-𑌜𑍍𑌞𑌮𑍁\-\ul{𑌪𑍇}\-𑌮𑌾𑌂 \ul{𑌦𑍇}\-𑌵𑍋 \ul{𑌦𑍇}\-𑌵𑌹𑍂᳴𑌤𑌿𑌮𑌵\-\ul{𑌤𑍁} 𑌵𑍇𑌤𑍍𑌵𑌾𑌽𑌽𑌜𑍍𑌯᳴\-\ul{𑌸𑍍𑌯} 𑌹𑍋\-\ul{𑌤}\-𑌰𑍍𑌯𑌜᳴।
𑌹𑍋𑌤𑌾᳴ 𑌯𑌕𑍍𑌷\-\ul{𑌦𑍍𑌬}\-𑌰𑍍‌॒\mbox{}𑌹𑌿𑌃 \ul{𑌸𑍁}\-𑌷𑍍𑌟\-\ul{𑌰𑍀}\-𑌮𑍋𑌰𑍍𑌣᳴𑌮𑍍𑌰𑌦𑌾 \ul{𑌅}\-𑌸𑍍𑌮𑌿𑌨𑍍 \ul{𑌯}\-𑌜𑍍𑌞𑍇 𑌵𑌿 \ul{𑌚} 𑌪𑍍𑌰 𑌚᳴ 𑌪𑍍𑌰𑌥𑌤𑌾𑍟 𑌸𑍍𑌵𑌾\-\ul{𑌸}\-𑌸𑍍𑌥𑌂 \ul{𑌦𑍇}\-𑌵𑍇𑌭𑍍𑌯𑌃᳴।
𑌏𑌮𑍇᳴𑌨\-\ul{𑌦}\-𑌦𑍍𑌯 𑌵𑌸᳴𑌵𑍋 \ul{𑌰𑍁}\-𑌦𑍍𑌰𑌾 𑌆᳴\-\ul{𑌦𑌿}\-𑌤𑍍𑌯𑌾𑌃 𑌸᳴𑌦𑌨𑍍𑌤𑍁 \ul{𑌪𑍍𑌰𑌿}\-𑌯𑌮𑌿𑌨𑍍𑌦𑍍𑌰᳴𑌸𑍍𑌯𑌾\-\ul{𑌸𑍍𑌤𑍁} 𑌵𑍇𑌤𑍍𑌵𑌾𑌽𑌽𑌜𑍍𑌯᳴\-\ul{𑌸𑍍𑌯} 𑌹𑍋\-\ul{𑌤}\-𑌰𑍍𑌯𑌜᳴॥4॥

%3.6.2.2
𑌹𑍋𑌤𑌾᳴ 𑌯\-\ul{𑌕𑍍𑌷}\-𑌦𑍍𑌦𑍁𑌰᳴ \ul{𑌋}\-𑌷𑍍𑌵𑌾𑌃 𑌕᳴\-\ul{𑌵}\-𑌷𑍍𑌯𑍋 𑌕𑍋᳴𑌷𑌧𑌾𑌵\-\ul{𑌨𑍀}\-𑌰𑍁𑌦𑌾𑌤𑌾᳴\-\ul{𑌭𑍀}\-𑌰𑍍𑌜𑌿𑌹᳴\-\ul{𑌤𑌾𑌂} 𑌵𑌿𑌪𑌕𑍍𑌷𑍋᳴𑌭𑌿𑌃 𑌶𑍍𑌰𑌯𑌨𑍍𑌤𑌾𑌮𑍍।
\-\ul{𑌸𑍁}\-\-\ul{𑌪𑍍𑌰𑌾}\-\-\ul{𑌯}\-𑌣𑌾 \ul{𑌅}\-𑌸𑍍𑌮𑌿𑌨𑍍 \ul{𑌯}\-𑌜𑍍𑌞𑍇 𑌵𑌿𑌶𑍍𑌰᳴𑌯𑌨𑍍𑌤𑌾𑌮𑍃\-\ul{𑌤𑌾}\-𑌵𑍃𑌧𑍋᳴ \ul{𑌵𑌿}\-𑌯𑌨𑍍𑌤𑍍𑌵𑌾𑌜𑍍𑌯᳴\-\ul{𑌸𑍍𑌯} 𑌹𑍋\-\ul{𑌤}\-𑌰𑍍𑌯𑌜᳴।
𑌹𑍋𑌤𑌾᳴ 𑌯𑌕𑍍𑌷\-\ul{𑌦𑍁}\-𑌷𑌾\-\ul{𑌸𑌾}\-𑌨𑌕𑍍𑌤𑌾᳴ 𑌬𑍃\-\ul{𑌹}\-𑌤𑍀 \ul{𑌸𑍁}\-𑌪𑍇𑌶᳴\-\ul{𑌸𑌾} 𑌨𑍄𑍟𑌃 𑌪𑌤𑌿᳴\-\ul{𑌭𑍍𑌯𑍋} 𑌯𑍋𑌨𑌿𑌂᳴ 𑌕𑍃\-\ul{𑌣𑍍𑌵𑌾}\-𑌨𑍇।
\-\ul{𑌸}\-\-\ul{𑍟}\-𑌸𑍍𑌮𑌯᳴𑌮𑌾\-\ul{𑌨𑍇} 𑌇𑌨𑍍𑌦𑍍𑌰𑍇᳴𑌣 \ul{𑌦𑍇}\-𑌵𑍈𑌰𑍇𑌦𑌂 \ul{𑌬}\-\-\ul{𑌰𑍍}\-𑌹𑌿𑌃 𑌸𑍀᳴𑌦𑌤𑌾𑌂 \ul{𑌵𑍀}\-𑌤𑌾𑌮𑌾𑌜𑍍𑌯᳴\-\ul{𑌸𑍍𑌯} 𑌹𑍋\-\ul{𑌤}\-𑌰𑍍𑌯𑌜᳴।
𑌹𑍋𑌤𑌾᳴ 𑌯\-\ul{𑌕𑍍𑌷}\-𑌦𑍍𑌦𑍈\-\ul{𑌵𑍍𑌯𑌾} 𑌹𑍋𑌤𑌾᳴𑌰𑌾 \ul{𑌮}\-𑌨𑍍𑌦𑍍𑌰𑌾 𑌪𑍋𑌤𑌾᳴𑌰𑌾 \ul{𑌕}\-𑌵𑍀 𑌪𑍍𑌰𑌚𑍇᳴𑌤𑌸𑌾।
𑌸𑍍𑌵𑌿᳴𑌷𑍍𑌟\-\ul{𑌮}\-𑌦𑍍𑌯𑌾𑌨𑍍𑌯𑌃 𑌕᳴𑌰\-\ul{𑌦𑌿}\-𑌷𑌾 𑌸𑍍𑌵᳴𑌭𑌿𑌗𑍂𑌰𑍍𑌤\-\ul{𑌮}\-𑌨𑍍𑌯 \ul{𑌊}\-𑌰𑍍𑌜𑌾 𑌸𑌤᳴𑌵\-\ul{𑌸𑍇}\-𑌮𑌂 \ul{𑌯}\-𑌜𑍍𑌞𑌂 \ul{𑌦𑌿}\-𑌵𑌿 \ul{𑌦𑍇}\-𑌵𑍇𑌷𑍁᳴ 𑌧𑌤𑍍𑌤𑌾𑌂 \ul{𑌵𑍀}\-𑌤𑌾𑌮𑌾𑌜𑍍𑌯᳴\-\ul{𑌸𑍍𑌯} 𑌹𑍋\-\ul{𑌤}\-𑌰𑍍𑌯𑌜᳴।
𑌹𑍋𑌤𑌾᳴ 𑌯𑌕𑍍𑌷\-\ul{𑌤𑍍𑌤𑌿}\-𑌸𑍍𑌰𑍋 \ul{𑌦𑍇}\-𑌵𑍀\-\ul{𑌰}\-𑌪𑌸𑌾᳴\-\ul{𑌮}\-𑌪𑌸𑍍𑌤᳴\-\ul{𑌮𑌾} 𑌅𑌚𑍍𑌛𑌿᳴𑌦𑍍𑌰\-\ul{𑌮}\-𑌦𑍍𑌯𑍇𑌦𑌮𑌪᳴𑌸𑍍𑌤𑌨𑍍𑌵𑌤𑌾𑌮𑍍।
\-\ul{𑌦𑍇}\-𑌵𑍇𑌭𑍍𑌯𑍋᳴ \ul{𑌦𑍇}\-𑌵𑍀\-\ul{𑌰𑍍𑌦𑍇}\-𑌵𑌮𑌪𑍋᳴ \ul{𑌵𑌿}\-𑌯𑌨𑍍𑌤𑍍𑌵𑌾𑌜𑍍𑌯᳴\-\ul{𑌸𑍍𑌯} 𑌹𑍋\-\ul{𑌤}\-𑌰𑍍𑌯𑌜᳴।
𑌹𑍋𑌤𑌾᳴ 𑌯\-\ul{𑌕𑍍𑌷}\-𑌤𑍍𑌤𑍍𑌵𑌷𑍍𑌟𑌾᳴\-\ul{𑌰}\-𑌮𑌚𑌿᳴\-\ul{𑌷𑍍𑌟𑍁}\-𑌮𑌪𑌾᳴𑌕𑍞 𑌰𑍇\-\ul{𑌤𑍋}\-𑌧𑌾𑌂 𑌵𑌿𑌶𑍍𑌰᳴𑌵𑌸𑌂 𑌯\-\ul{𑌶𑍋}\-𑌧𑌾𑌮𑍍।
\-\ul{𑌪𑍁}\-\-\ul{𑌰𑍁}\-𑌰𑍂\-\ul{𑌪}\-𑌮𑌕𑌾᳴𑌮𑌕𑌰𑍍‌\mbox{}𑌶𑌨𑍞 \ul{𑌸𑍁}\-𑌪𑍋\-\ul{𑌷𑌃} 𑌪𑍋\-\ul{𑌷𑍈𑌃} 𑌸𑍍𑌯𑌾\-\ul{𑌥𑍍𑌸𑍁}\-𑌵𑍀𑌰𑍋᳴ \ul{𑌵𑍀}\-𑌰𑍈𑌰𑍍𑌵𑍇𑌤𑍍𑌵𑌾𑌽𑌽𑌜𑍍𑌯᳴\-\ul{𑌸𑍍𑌯} 𑌹𑍋\-\ul{𑌤}\-𑌰𑍍𑌯𑌜᳴।
𑌹𑍋𑌤𑌾᳴ 𑌯\-\ul{𑌕𑍍𑌷}\-𑌦𑍍𑌵\-\ul{𑌨}\-𑌸𑍍𑌪𑌤𑌿᳴\-\ul{𑌮𑍁}\-𑌪𑌾𑌵᳴𑌸𑍍𑌰𑌕𑍍𑌷\-\ul{𑌦𑍍𑌧𑌿}\-𑌯𑍋 \ul{𑌜𑍋}\-𑌷𑍍𑌟𑌾𑌰𑍞᳴ \ul{𑌶}\-𑌶\-\ul{𑌮}\-𑌨𑍍𑌨𑌰𑌃᳴।
𑌸𑍍𑌵\-\ul{𑌦𑌾}\-𑌥𑍍𑌸𑍍𑌵𑌧𑌿᳴𑌤𑌿𑌰𑍍\mbox{}𑌋\-\ul{𑌤𑍁}\-𑌥𑌾𑌦𑍍𑌯 \ul{𑌦𑍇}\-𑌵𑍋 \ul{𑌦𑍇}\-𑌵𑍇𑌭𑍍𑌯𑍋᳴ \ul{𑌹}\-𑌵𑍍𑌯𑌾\-\ul{𑌵𑌾}\-𑌡𑍍𑌵𑍇𑌤𑍍𑌵𑌾𑌽𑌽𑌜𑍍𑌯᳴\-\ul{𑌸𑍍𑌯} 𑌹𑍋\-\ul{𑌤}\-𑌰𑍍𑌯𑌜᳴।
𑌹𑍋𑌤𑌾᳴ 𑌯𑌕𑍍𑌷\-\ul{𑌦}\-𑌗𑍍𑌨𑌿𑍟 𑌸𑍍𑌵𑌾𑌹𑌾\-𑌽𑌽𑌜𑍍𑌯᳴\-\ul{𑌸𑍍𑌯} 𑌸𑍍𑌵𑌾\-\ul{𑌹𑌾} 𑌮𑍇𑌦᳴\-\ul{𑌸𑌃} 𑌸𑍍𑌵𑌾𑌹𑌾॑ \ul{𑌸𑍍𑌤𑍋}\-𑌕𑌾\-\ul{𑌨𑌾}\-\-\ul{𑍟} 𑌸𑍍𑌵𑌾\-\ul{𑌹𑌾} 𑌸𑍍𑌵𑌾𑌹𑌾᳴𑌕𑍃𑌤𑍀\-\ul{𑌨𑌾}\-\-\ul{𑍟} 𑌸𑍍𑌵𑌾𑌹𑌾᳴ \ul{𑌹}\-𑌵𑍍𑌯𑌸𑍂॑𑌕𑍍𑌤𑍀𑌨𑌾𑌮𑍍।
𑌸𑍍𑌵𑌾𑌹𑌾᳴ \ul{𑌦𑍇}\-𑌵𑌾𑍞 𑌆॑\-\ul{𑌜𑍍𑌯}\-𑌪𑌾𑌨𑍍𑌥𑍍𑌸𑍍𑌵𑌾\-\ul{𑌹𑌾}\-\-𑌽𑌗𑍍𑌨𑌿𑍞 \ul{𑌹𑍋}\-𑌤𑍍𑌰𑌾𑌜𑍍𑌜𑍁᳴\-\ul{𑌷𑌾}\-𑌣𑌾 𑌅\-\ul{𑌗𑍍𑌨} 𑌆𑌜𑍍𑌯᳴𑌸𑍍𑌯 𑌵𑌿𑌯\-\ul{𑌨𑍍𑌤𑍁} 𑌹𑍋\-\ul{𑌤}\-𑌰𑍍𑌯𑌜᳴॥5॥\anuvakamend[\-\ul{𑌪𑍍𑌰𑌿}\-𑌯𑌮𑌿𑌨𑍍𑌦𑍍𑌰᳴𑌸𑍍𑌯𑌾\-\ul{𑌸𑍍𑌤𑍁} 𑌵𑍇𑌤𑍍𑌵𑌾𑌽𑌽𑌜𑍍𑌯᳴\-\ul{𑌸𑍍𑌯} 𑌹𑍋\-\ul{𑌤}\-𑌰𑍍𑌯𑌜᳴ \ul{𑌸𑍁}\-𑌵𑍀𑌰𑍋᳴ \ul{𑌵𑍀}\-𑌰𑍈𑌰𑍍𑌵𑍇𑌤𑍍𑌵𑌾𑌽𑌽𑌜𑍍𑌯᳴\-\ul{𑌸𑍍𑌯} 𑌹𑍋\-\ul{𑌤}\-𑌰𑍍𑌯𑌜᳴ \ul{𑌚}\-𑌤𑍍𑌵𑌾𑌰𑌿᳴ 𑌚 (\-\ul{𑌅}\-𑌗𑍍𑌨𑌿𑌨𑍍𑌤\-\ul{𑌨𑍂}\-𑌨𑌪𑌾᳴\-\ul{𑌤}\-𑌨𑍍𑌨\-\ul{𑌰𑌾}\-𑌶𑍞𑌸᳴\-\ul{𑌮}\-𑌗𑍍𑌨𑌿\-\ul{𑌮𑌿}\-𑌡 𑌈᳴\-\ul{𑌡𑌿}\-𑌤𑍋 \ul{𑌬}\-𑌰𑍍‌॒\mbox{}𑌹𑌿𑌰𑍍𑌦𑍁𑌰᳴ \ul{𑌉}\-𑌷𑌾\-\ul{𑌸𑌾}\-𑌨\-\ul{𑌕𑍍𑌤𑌾} 𑌦𑍈𑌵𑍍𑌯𑌾᳴ \ul{𑌤𑌿}\-𑌸𑍍𑌰𑌸𑍍𑌤𑍍𑌵𑌷𑍍𑌟𑌾᳴\-\ul{𑌰𑌂} 𑌵\-\ul{𑌨}\-𑌸𑍍𑌪𑌤𑌿᳴\-\ul{𑌮}\-𑌗𑍍𑌨𑌿𑌮𑍍।
𑌪\-\ul{𑌞𑍍𑌚} 𑌵𑍇𑌤𑍍𑌵𑍇𑌕𑍋᳴ \ul{𑌵𑌿}\-𑌯\-\ul{𑌨𑍍𑌤𑍁} 𑌦𑍍𑌵𑌿\-\ul{𑌰𑍍𑌵𑍀}\-𑌤𑌾𑌮𑍇𑌕𑍋᳴ \ul{𑌵𑌿}\-𑌯\-\ul{𑌨𑍍𑌤𑍁} 𑌦𑍍𑌵𑌿𑌰𑍍𑌵𑍇𑌤𑍍𑌵𑍇𑌕𑍋᳴ 𑌵𑌿𑌯\-\ul{𑌨𑍍𑌤𑍁} 𑌹𑍋\-\ul{𑌤}\-𑌰𑍍𑌯𑌜᳴॥)]

%3.6.3.1
𑌸𑌮𑌿᳴𑌦𑍍𑌧𑍋 \ul{𑌅}\-𑌦𑍍𑌯 𑌮𑌨𑍁᳴𑌷𑍋 𑌦𑍁\-\ul{𑌰𑍋}\-𑌣𑍇।
\-\ul{𑌦𑍇}\-𑌵𑍋 \ul{𑌦𑍇}\-𑌵𑌾𑌨𑍍 𑌯᳴𑌜𑌸𑌿 𑌜𑌾𑌤𑌵𑍇𑌦𑌃।
𑌆 \ul{𑌚} 𑌵𑌹᳴ 𑌮𑌿𑌤𑍍𑌰𑌮𑌹𑌶𑍍𑌚𑌿\-\ul{𑌕𑌿}\-𑌤𑍍𑌵𑌾𑌨𑍍।
𑌤𑍍𑌵𑌂 \ul{𑌦𑍂}\-𑌤𑌃 \ul{𑌕}\-𑌵𑌿𑌰᳴\-\ul{𑌸𑌿} 𑌪𑍍𑌰𑌚𑍇᳴𑌤𑌾𑌃।
𑌤𑌨𑍂᳴𑌨𑌪𑌾\-\ul{𑌤𑍍𑌪}\-𑌥 \ul{𑌋}\-𑌤\-\ul{𑌸𑍍𑌯} 𑌯𑌾𑌨𑌾𑌨𑍍᳴।
𑌮𑌧𑍍𑌵𑌾᳴ 𑌸\-\ul{𑌮}\-𑌞𑍍𑌜𑌨𑍍𑌥𑍍𑌸𑍍𑌵᳴𑌦𑌯𑌾 𑌸𑍁𑌜𑌿𑌹𑍍𑌵।
𑌮𑌨𑍍𑌮𑌾᳴𑌨𑌿 \ul{𑌧𑍀}\-𑌭𑌿\-\ul{𑌰𑍁}\-𑌤 \ul{𑌯}\-𑌜𑍍𑌞\-\ul{𑌮𑍃}\-𑌨𑍍𑌧𑌨𑍍।
\-\ul{𑌦𑍇}\-\-\ul{𑌵}\-𑌤𑍍𑌰𑌾 𑌚᳴ 𑌕𑍃𑌣𑍁𑌹𑍍𑌯\-\ul{𑌧𑍍𑌵}\-𑌰𑌂 𑌨𑌃᳴।
𑌨\-\ul{𑌰𑌾}\-𑌶𑍞𑌸᳴𑌸𑍍𑌯 𑌮\-\ul{𑌹𑌿}\-𑌮𑌾𑌨᳴𑌮𑍇𑌷𑌾𑌮𑍍।
𑌉𑌪᳴ 𑌸𑍍𑌤𑍋𑌷𑌾𑌮 𑌯\-\ul{𑌜}\-𑌤𑌸𑍍𑌯᳴ \ul{𑌯}\-𑌜𑍍𑌞𑍈𑌃॥6॥

%3.6.3.2
𑌤𑍇 \ul{𑌸𑍁}\-𑌕𑍍𑌰𑌤᳴\-\ul{𑌵𑌃} 𑌶𑍁𑌚᳴𑌯𑍋 𑌧𑌿\-\ul{𑌯}\-𑌨𑍍𑌧𑌾𑌃।
𑌸𑍍𑌵𑌦᳴𑌨𑍍𑌤𑍁 \ul{𑌦𑍇}\-𑌵𑌾 \ul{𑌉}\-𑌭𑌯𑌾᳴𑌨𑌿 \ul{𑌹}\-𑌵𑍍𑌯𑌾।
\-\ul{𑌆}\-𑌜𑍁𑌹𑍍𑌵𑌾᳴\-\ul{𑌨} 𑌈\-\ul{𑌡𑍍𑌯𑍋} 𑌵𑌨𑍍𑌦𑍍𑌯᳴𑌶𑍍𑌚।
𑌆𑌯𑌾॑𑌹𑍍𑌯\-\ul{𑌗𑍍𑌨𑍇} 𑌵𑌸𑍁᳴𑌭𑌿𑌃 \ul{𑌸}\-𑌜𑍋𑌷𑌾𑌃॑।
𑌤𑍍𑌵𑌂 \ul{𑌦𑍇}\-𑌵𑌾𑌨𑌾᳴𑌮𑌸𑌿 𑌯\-\ul{𑌹𑍍𑌵} 𑌹𑍋𑌤𑌾॑।
𑌸 𑌏᳴𑌨𑌾𑌨𑍍 𑌯𑌕𑍍𑌷𑍀\-\ul{𑌷𑌿}\-𑌤𑍋 𑌯𑌜𑍀᳴𑌯𑌾𑌨𑍍।
\-\ul{𑌪𑍍𑌰𑌾}\-𑌚𑍀𑌨𑌂᳴ \ul{𑌬}\-𑌰𑍍‌॒\mbox{}𑌹𑌿𑌃 \ul{𑌪𑍍𑌰}\-𑌦𑌿𑌶𑌾᳴ 𑌪𑍃\-\ul{𑌥𑌿}\-𑌵𑍍𑌯𑌾𑌃।
𑌵𑌸𑍍𑌤𑍋᳴\-\ul{𑌰}\-𑌸𑍍𑌯𑌾 𑌵𑍃᳴𑌜𑍍𑌯\-\ul{𑌤𑍇} 𑌅\-\ul{𑌗𑍍𑌰𑍇} 𑌅𑌹𑍍𑌨𑌾॑𑌮𑍍।
𑌵𑍍𑌯𑍁᳴ 𑌪𑍍𑌰𑌥𑌤𑍇 𑌵𑌿\-\ul{𑌤}\-𑌰𑌂 𑌵𑌰𑍀᳴𑌯𑌃।
\-\ul{𑌦𑍇}\-𑌵𑍇\-\ul{𑌭𑍍𑌯𑍋} 𑌅𑌦𑌿᳴𑌤𑌯𑍇 \ul{𑌸𑍍𑌯𑍋}\-𑌨𑌮𑍍॥7॥

%3.6.3.3
𑌵𑍍𑌯𑌚᳴𑌸𑍍𑌵𑌤𑍀𑌰𑍁\-\ul{𑌰𑍍𑌵𑌿}\-𑌯𑌾 𑌵𑌿𑌶𑍍𑌰᳴𑌯𑌨𑍍𑌤𑌾𑌮𑍍।
𑌪𑌤𑌿᳴\-\ul{𑌭𑍍𑌯𑍋} 𑌨 𑌜𑌨᳴\-\ul{𑌯𑌃} 𑌶𑍁𑌮𑍍𑌭᳴𑌮𑌾𑌨𑌾𑌃।
𑌦𑍇𑌵𑍀॑𑌰𑍍𑌦𑍍𑌵𑌾𑌰𑍋 𑌬𑍃𑌹𑌤𑍀𑌰𑍍𑌵𑌿𑌶𑍍𑌵𑌮𑌿𑌨𑍍𑌵𑌾𑌃।
\-\ul{𑌦𑍇}\-𑌵𑍇𑌭𑍍𑌯𑍋᳴ 𑌭𑌵𑌥 𑌸𑍁𑌪𑍍𑌰𑌾\-\ul{𑌯}\-𑌣𑌾𑌃।
𑌆\-\ul{𑌸𑍁}\-𑌷𑍍𑌵𑌯᳴𑌨𑍍𑌤𑍀 𑌯\-\ul{𑌜}\-𑌤𑍇 𑌉𑌪𑌾᳴𑌕𑍇।
\-\ul{𑌉}\-𑌷𑌾\-\ul{𑌸𑌾}\-𑌨𑌕𑍍𑌤𑌾᳴ 𑌸𑌦\-\ul{𑌤𑌾𑌂} 𑌨𑌿 𑌯𑍋𑌨𑍗॑।
\-\ul{𑌦𑌿}\-𑌵𑍍𑌯𑍇 𑌯𑍋𑌷᳴𑌣𑍇 𑌬𑍃\-\ul{𑌹}\-𑌤𑍀 𑌸𑍁᳴\-\ul{𑌰𑍁}\-𑌕𑍍𑌮𑍇।
𑌅\-\ul{𑌧𑌿} 𑌶𑍍𑌰𑌿𑌯𑍞᳴ 𑌶𑍁\-\ul{𑌕𑍍𑌰}\-𑌪𑌿\-\ul{𑌶𑌂} 𑌦𑌧𑌾᳴𑌨𑍇।
𑌦𑍈\-\ul{𑌵𑍍𑌯𑌾} 𑌹𑍋𑌤𑌾᳴𑌰𑌾 𑌪𑍍𑌰\-\ul{𑌥}\-𑌮𑌾 \ul{𑌸𑍁}\-𑌵𑌾𑌚𑌾॑।
𑌮𑌿𑌮𑌾᳴𑌨𑌾 \ul{𑌯}\-𑌜𑍍𑌞𑌂 𑌮𑌨𑍁᳴\-\ul{𑌷𑍋} 𑌯𑌜᳴𑌧𑍍𑌯𑍈॥8॥

%3.6.3.4
\-\ul{𑌪𑍍𑌰}\-\-\ul{𑌚𑍋}\-𑌦𑌯᳴𑌨𑍍𑌤𑌾 \ul{𑌵𑌿}\-𑌦𑌥𑍇᳴𑌷𑍁 \ul{𑌕𑌾}\-𑌰𑍂।
\-\ul{𑌪𑍍𑌰𑌾}\-𑌚𑍀\-\ul{𑌨𑌂} 𑌜𑍍𑌯𑍋𑌤𑌿𑌃᳴ \ul{𑌪𑍍𑌰}\-𑌦𑌿𑌶𑌾᳴ \ul{𑌦𑌿}\-𑌶𑌨𑍍𑌤𑌾॑।
𑌆 𑌨𑍋᳴ \ul{𑌯}\-𑌜𑍍𑌞𑌂 𑌭𑌾𑌰᳴\-\ul{𑌤𑍀} 𑌤𑍂𑌯᳴𑌮𑍇𑌤𑍁।
𑌇𑌡𑌾᳴ 𑌮\-\ul{𑌨𑍁}\-𑌷𑍍𑌵\-\ul{𑌦𑌿}\-𑌹 \ul{𑌚𑍇}\-𑌤𑌯᳴𑌨𑍍𑌤𑍀।
\-\ul{𑌤𑌿}\-𑌸𑍍𑌰𑍋 \ul{𑌦𑍇}\-𑌵𑍀\-\ul{𑌰𑍍𑌬}\-𑌰𑍍‌॒\mbox{}𑌹𑌿𑌰𑍇𑌦𑍟 \ul{𑌸𑍍𑌯𑍋}\-𑌨𑌮𑍍।
𑌸𑌰᳴𑌸𑍍𑌵\-\ul{𑌤𑍀} 𑌸𑍍𑌵𑌪᳴𑌸𑌃 𑌸𑌦𑌨𑍍𑌤𑍁।
𑌯 \ul{𑌇}\-𑌮𑍇 𑌦𑍍𑌯𑌾𑌵𑌾᳴𑌪𑍃\-\ul{𑌥𑌿}\-𑌵𑍀 𑌜𑌨𑌿᳴𑌤𑍍𑌰𑍀।
\-\ul{𑌰𑍂}\-𑌪𑍈𑌰𑌪𑌿𑍞᳴\-\ul{𑌶}\-𑌦𑍍𑌭𑍁𑌵᳴𑌨𑌾\-\ul{𑌨𑌿} 𑌵𑌿𑌶𑍍𑌵𑌾॑।
𑌤\-\ul{𑌮}\-𑌦𑍍𑌯 𑌹𑍋᳴𑌤𑌰𑌿\-\ul{𑌷𑌿}\-𑌤𑍋 𑌯𑌜𑍀᳴𑌯𑌾𑌨𑍍।
\-\ul{𑌦𑍇}\-𑌵𑌂 𑌤𑍍𑌵𑌷𑍍𑌟𑌾᳴𑌰\-\ul{𑌮𑌿}\-𑌹 𑌯᳴𑌕𑍍𑌷𑌿 \ul{𑌵𑌿}\-𑌦𑍍𑌵𑌾𑌨𑍍॥9॥

%3.6.3.5
\-\ul{𑌉}\-𑌪𑌾𑌵᳴𑌸𑍃\-\ul{𑌜}\-𑌤𑍍𑌮𑌨𑍍𑌯𑌾᳴ 𑌸\-\ul{𑌮}\-𑌞𑍍𑌜𑌨𑍍।
\-\ul{𑌦𑍇}\-𑌵𑌾\-\ul{𑌨𑌾𑌂} 𑌪𑌾𑌥᳴ 𑌋\-\ul{𑌤𑍁}\-𑌥𑌾 \ul{𑌹}\-𑌵𑍀𑍞𑌷𑌿᳴।
𑌵\-\ul{𑌨}\-𑌸𑍍𑌪𑌤𑌿𑌃᳴ 𑌶\-\ul{𑌮𑌿}\-𑌤𑌾 \ul{𑌦𑍇}\-𑌵𑍋 \ul{𑌅}\-𑌗𑍍𑌨𑌿𑌃।
𑌸𑍍𑌵𑌦᳴𑌨𑍍𑌤𑍁 \ul{𑌹}\-𑌵𑍍𑌯𑌂 𑌮𑌧𑍁᳴𑌨𑌾 \ul{𑌘𑍃}\-𑌤𑍇𑌨᳴।
\-\ul{𑌸}\-𑌦𑍍𑌯𑍋 \ul{𑌜𑌾}\-𑌤𑍋 𑌵𑍍𑌯᳴𑌮𑌿𑌮𑍀𑌤 \ul{𑌯}\-𑌜𑍍𑌞𑌮𑍍।
\-\ul{𑌅}\-𑌗𑍍𑌨𑌿\-\ul{𑌰𑍍𑌦𑍇}\-𑌵𑌾𑌨𑌾᳴𑌮𑌭𑌵𑌤𑍍𑌪𑍁\-\ul{𑌰𑍋}\-𑌗𑌾𑌃।
\-\ul{𑌅}\-𑌸𑍍𑌯 𑌹𑍋𑌤𑍁𑌃᳴ \ul{𑌪𑍍𑌰}\-𑌦𑌿\-\ul{𑌶𑍍𑌯𑍃}\-𑌤𑌸𑍍𑌯᳴ \ul{𑌵𑌾}\-𑌚𑌿।
𑌸𑍍𑌵𑌾𑌹𑌾᳴𑌕𑍃𑌤𑍞 \ul{𑌹}\-𑌵𑌿𑌰᳴𑌦𑌨𑍍𑌤𑍁 \ul{𑌦𑍇}\-𑌵𑌾𑌃॥10॥\anuvakamend[\-\ul{𑌯}\-𑌜𑍍𑌞𑍈𑌃 \ul{𑌸𑍍𑌯𑍋}\-𑌨𑌂 𑌯𑌜᳴𑌧𑍍𑌯𑍈 \ul{𑌵𑌿}\-𑌦𑍍𑌵𑌾\-\ul{𑌨}\-𑌷𑍍𑌟𑍗 𑌚᳴]

%3.6.4.1
\-\ul{𑌅}\-𑌗𑍍𑌨𑌿𑌰𑍍‌\mbox{}𑌹𑍋𑌤𑌾᳴ 𑌨𑍋 𑌅\-\ul{𑌧𑍍𑌵}\-𑌰𑍇।
\-\ul{𑌵𑌾}\-𑌜𑍀 𑌸𑌨𑍍𑌪𑌰𑌿᳴𑌣𑍀𑌯𑌤𑍇।
\-\ul{𑌦𑍇}\-𑌵𑍋 \ul{𑌦𑍇}\-𑌵𑍇𑌷𑍁᳴ \ul{𑌯}\-𑌜𑍍𑌞𑌿𑌯𑌃᳴।
𑌪𑌰𑌿᳴𑌤𑍍𑌰𑌿\-\ul{𑌵𑌿}\-𑌷𑍍𑌟𑍍𑌯᳴\-\ul{𑌧𑍍𑌵}\-𑌰𑌮𑍍।
𑌯𑌾\-\ul{𑌤𑍍𑌯}\-𑌗𑍍𑌨𑍀 \ul{𑌰}\-𑌥𑍀𑌰𑌿᳴𑌵।
𑌆 \ul{𑌦𑍇}\-𑌵𑍇\-\ul{𑌷𑍁} 𑌪𑍍𑌰\-\ul{𑌯𑍋} 𑌦𑌧᳴𑌤𑍍।
𑌪\-\ul{𑌰𑌿} 𑌵𑌾𑌜᳴𑌪𑌤𑌿𑌃 \ul{𑌕}\-𑌵𑌿𑌃।
\-\ul{𑌅}\-𑌗𑍍𑌨𑌿𑌰𑍍‌\mbox{}\-\ul{𑌹}\-𑌵𑍍𑌯𑌾𑌨𑍍𑌯᳴𑌕𑍍𑌰𑌮𑍀𑌤𑍍।
𑌦\-\ul{𑌧}\-𑌦𑍍𑌰𑌤𑍍𑌨𑌾᳴𑌨𑌿 \ul{𑌦𑌾}\-𑌶𑍁𑌷𑍇॑॥11॥\anuvakamend[\-\ul{𑌅}\-𑌗𑍍𑌨𑌿𑌰𑍍‌\mbox{}𑌹𑍋𑌤𑌾᳴ \ul{𑌨𑍋} 𑌨𑌵᳴]

%3.6.5.1
𑌅𑌜𑍈᳴\-\ul{𑌦}\-𑌗𑍍𑌨𑌿𑌃।
𑌅𑌸᳴\-\ul{𑌨}\-𑌦𑍍𑌵𑌾\-\ul{𑌜}\-𑌨𑍍𑌨𑌿।
\-\ul{𑌦𑍇}\-𑌵𑍋 \ul{𑌦𑍇}\-𑌵𑍇𑌭𑍍𑌯𑍋᳴ \ul{𑌹}\-𑌵𑍍𑌯𑌾𑌵𑌾॑𑌟𑍍।
𑌪𑍍𑌰𑌾𑌞𑍍𑌜𑍋᳴𑌭𑌿𑌰𑍍‌\mbox{}𑌹𑌿\-\ul{𑌨𑍍𑌵𑌾}\-𑌨𑌃।
𑌧𑍇𑌨𑌾᳴\-\ul{𑌭𑌿𑌃} 𑌕𑌲𑍍𑌪᳴𑌮𑌾𑌨𑌃।
\-\ul{𑌯}\-𑌜𑍍𑌞𑌸𑍍𑌯𑌾𑌯𑍁𑌃᳴ 𑌪𑍍𑌰\-\ul{𑌤𑌿}\-𑌰𑌨𑍍।
𑌉\-\ul{𑌪} 𑌪𑍍𑌰𑍇𑌷𑍍𑌯᳴ 𑌹𑍋𑌤𑌃।
\-\ul{𑌹}\-𑌵𑍍𑌯𑌾 \ul{𑌦𑍇}\-𑌵𑍇𑌭𑍍𑌯𑌃᳴॥12॥\anuvakamend[𑌅𑌜𑍈᳴\-\ul{𑌦}\-𑌷𑍍𑌟𑍗]

%3.6.6.1
𑌦𑍈𑌵𑍍𑌯𑌾𑌃॑ 𑌶𑌮𑌿𑌤𑌾𑌰 \ul{𑌉}\-𑌤 𑌮᳴𑌨𑍁\-\ul{𑌷𑍍𑌯𑌾} 𑌆𑌰᳴𑌭𑌧𑍍𑌵𑌮𑍍।
𑌉𑌪᳴𑌨𑌯\-\ul{𑌤} 𑌮𑍇\-\ul{𑌧𑍍𑌯𑌾} 𑌦𑍁𑌰𑌃᳴।
\-\ul{𑌆}\-𑌶𑌾𑌸𑌾᳴\-\ul{𑌨𑌾} 𑌮𑍇𑌧᳴𑌪𑌤𑌿\-\ul{𑌭𑍍𑌯𑌾𑌂} 𑌮𑍇𑌧𑌮𑍍॑।
𑌪𑍍𑌰𑌾𑌸𑍍𑌮𑌾᳴ \ul{𑌅}\-𑌗𑍍𑌨𑌿𑌂 𑌭᳴𑌰𑌤।
\-\ul{𑌸𑍍𑌤𑍃}\-\-\ul{𑌣𑍀}\-𑌤 \ul{𑌬}\-\-\ul{𑌰𑍍}\-𑌹𑌿𑌃।
𑌅𑌨𑍍𑌵𑍇᳴𑌨𑌂 \ul{𑌮𑌾}\-𑌤𑌾 𑌮᳴𑌨𑍍𑌯𑌤𑌾𑌮𑍍।
𑌅𑌨𑍁᳴ \ul{𑌪𑌿}\-𑌤𑌾।
𑌅\-\ul{𑌨𑍁} 𑌭𑍍𑌰𑌾\-\ul{𑌤𑌾} 𑌸𑌗᳴𑌰𑍍𑌭𑍍𑌯𑌃।
𑌅\-\ul{𑌨𑍁} 𑌸\-\ul{𑌖𑌾} 𑌸𑌯𑍂॑𑌥𑍍𑌯𑌃।
\-\ul{𑌉}\-\-\ul{𑌦𑍀}\-𑌚𑍀𑌨𑌾𑍞᳴ 𑌅𑌸𑍍𑌯 \ul{𑌪}\-𑌦𑍋 𑌨𑌿𑌧᳴𑌤𑍍𑌤𑌾𑌤𑍍॥13॥

%3.6.6.2
𑌸𑍂\-\ul{𑌰𑍍𑌯𑌂} 𑌚𑌕𑍍𑌷𑍁᳴𑌰𑍍𑌗𑌮𑌯𑌤𑌾𑌤𑍍।
𑌵𑌾𑌤𑌂᳴ \ul{𑌪𑍍𑌰𑌾}\-𑌣\-\ul{𑌮}\-𑌨𑍍𑌵𑌵᳴\-𑌸𑍃𑌜𑌤𑌾𑌤𑍍।
𑌦𑌿\-\ul{𑌶𑌃} 𑌶𑍍𑌰𑍋𑌤𑍍𑌰𑌮𑍍॑।
\-\ul{𑌅}\-𑌨𑍍𑌤𑌰𑌿᳴\-\ul{𑌕𑍍𑌷}\-𑌮𑌸𑍁𑌮𑍍॑।
\-\ul{𑌪𑍃}\-\-\ul{𑌥𑌿}\-𑌵𑍀𑍞 𑌶𑌰𑍀᳴𑌰𑌮𑍍।
\-\ul{𑌏}\-\-\ul{𑌕}\-𑌧𑌾\-𑌽\-\ul{𑌸𑍍𑌯} 𑌤𑍍𑌵\-\ul{𑌚}\-𑌮𑌾𑌚𑍍𑌛𑍍𑌯᳴𑌤𑌾𑌤𑍍।
\-\ul{𑌪𑍁}\-𑌰𑌾 𑌨𑌾𑌭𑍍𑌯𑌾᳴ 𑌅\-\ul{𑌪𑌿}\-𑌶𑌸𑍋᳴ \ul{𑌵}\-𑌪𑌾𑌮𑍁𑌤𑍍𑌖𑌿᳴𑌦𑌤𑌾𑌤𑍍।
\-\ul{𑌅}\-𑌨𑍍𑌤\-\ul{𑌰𑍇}\-𑌵𑍋𑌷𑍍𑌮𑌾𑌣𑌂᳴ 𑌵𑌾𑌰𑌯𑌤𑌾𑌤𑍍।
\-\ul{𑌶𑍍𑌯𑍇}\-𑌨𑌮᳴\-\ul{𑌸𑍍𑌯} 𑌵𑌕𑍍𑌷𑌃᳴ 𑌕𑍃𑌣𑍁𑌤𑌾𑌤𑍍।
\-\ul{𑌪𑍍𑌰}\-𑌶𑌸𑌾᳴ \ul{𑌬𑌾}\-𑌹𑍂॥14॥

%3.6.6.3
\-\ul{𑌶}\-𑌲𑌾 \ul{𑌦𑍋}\-𑌷𑌣𑍀॑।
\-\ul{𑌕}\-𑌶𑍍𑌯\-\ul{𑌪𑍇}\-𑌵𑌾𑍞𑌸𑌾॑।
𑌅𑌚𑍍𑌛𑌿᳴\-\ul{𑌦𑍍𑌰𑍇} 𑌶𑍍𑌰𑍋𑌣𑍀॑।
\-\ul{𑌕}\-𑌵\-\ul{𑌷𑍋}\-𑌰𑍂 \ul{𑌸𑍍𑌰𑍇}\-𑌕𑌪᳴𑌰𑍍𑌣𑌾\-\ul{𑌷𑍍𑌠𑍀}\-𑌵𑌨𑍍𑌤𑌾॑।
𑌷𑌡𑍍𑌵𑌿𑍞᳴𑌶𑌤𑌿𑌰\-\ul{𑌸𑍍𑌯} 𑌵𑌙𑍍𑌕𑍍𑌰᳴𑌯𑌃।
𑌤𑌾 𑌅᳴\-\ul{𑌨𑍁}\-𑌷𑍍𑌠𑍍𑌯𑍋𑌚𑍍𑌚𑍍𑌯𑌾᳴𑌵𑌯𑌤𑌾𑌤𑍍।
𑌗𑌾𑌤𑍍𑌰𑌂᳴ 𑌗𑌾𑌤𑍍𑌰\-\ul{𑌮}\-𑌸𑍍𑌯𑌾𑌨𑍂᳴𑌨𑌂 𑌕𑍃𑌣𑍁𑌤𑌾𑌤𑍍।
\-\ul{𑌊}\-\-\ul{𑌵}\-\-\ul{𑌧𑍍𑌯}\-\-\ul{𑌗𑍋}\-𑌹𑌂 𑌪𑌾𑌰𑍍𑌥𑌿᳴𑌵𑌂 𑌖𑌨𑌤𑌾𑌤𑍍।
\-\ul{𑌅}\-𑌸𑍍𑌨𑌾 𑌰\-\ul{𑌕𑍍𑌷𑌃} 𑌸𑍞𑌸𑍃᳴𑌜𑌤𑌾𑌤𑍍।
\-\ul{𑌵}\-\-\ul{𑌨𑌿}\-𑌷𑍍𑌠𑍁𑌮᳴\-\ul{𑌸𑍍𑌯} 𑌮𑌾 𑌰𑌾᳴𑌵𑌿𑌷𑍍𑌟॥15॥

%3.6.6.4
𑌉𑌰𑍂᳴\-\ul{𑌕𑌂} 𑌮𑌨𑍍𑌯᳴𑌮𑌾𑌨𑌾𑌃।
𑌨𑍇𑌦𑍍𑌵᳴\-\ul{𑌸𑍍𑌤𑍋}\-𑌕𑍇 𑌤𑌨᳴𑌯𑍇।
𑌰𑌵𑌿᳴\-\ul{𑌤𑌾}\-𑌰𑌵᳴𑌚𑍍𑌛𑌮𑌿𑌤𑌾𑌰𑌃।
𑌅𑌧𑍍𑌰𑌿᳴𑌗𑍋 𑌶\-\ul{𑌮𑍀}\-𑌧𑍍𑌵𑌮𑍍।
\-\ul{𑌸𑍁}\-𑌶𑌮𑌿᳴ 𑌶𑌮𑍀𑌧𑍍𑌵𑌮𑍍।
\-\ul{𑌶}\-\-\ul{𑌮𑍀}\-𑌧𑍍𑌵𑌮᳴𑌧𑍍𑌰𑌿𑌗𑍋।
𑌅𑌧𑍍𑌰𑌿᳴\-\ul{𑌗𑍁}\-𑌶𑍍𑌚𑌾𑌪𑌾᳴𑌪𑌶𑍍𑌚।
\-\ul{𑌉}\-𑌭𑍗 \ul{𑌦𑍇}\-𑌵𑌾𑌨𑌾𑍞᳴ 𑌶\-\ul{𑌮𑌿}\-𑌤𑌾𑌰𑍗॑।
𑌤𑌾\-\ul{𑌵𑌿}\-𑌮𑌂 \ul{𑌪}\-𑌶𑍂𑍟 𑌶𑍍𑌰᳴𑌪𑌯𑌤𑌾𑌂 𑌪𑍍𑌰\-\ul{𑌵𑌿}\-𑌦𑍍𑌵𑌾𑍞𑌸𑍗॑।
𑌯𑌥𑌾᳴𑌯𑌥𑌾\-𑌽\-\ul{𑌸𑍍𑌯} 𑌶𑍍𑌰𑌪᳴\-\ul{𑌣}\-𑌨𑍍𑌤𑌥𑌾᳴𑌤𑌥𑌾॥16॥\anuvakamend[\-\ul{𑌧}\-\-\ul{𑌤𑍍𑌤𑌾}\-\-\ul{𑌦𑍍𑌬𑌾}\-𑌹𑍂 𑌮𑌾 𑌰𑌾᳴𑌵𑌿\-\ul{𑌷𑍍𑌟} 𑌤𑌥𑌾᳴𑌤𑌥𑌾]

%3.6.7.1
\-\ul{𑌜𑍁}\-𑌷𑌸𑍍𑌵᳴ \ul{𑌸}\-𑌪𑍍𑌰𑌥᳴𑌸𑍍𑌤𑌮𑌮𑍍।
𑌵𑌚𑍋᳴ \ul{𑌦𑍇}\-𑌵𑌫𑍍𑌸᳴𑌰𑌸𑍍𑌤𑌮𑌮𑍍।
\-\ul{𑌹}\-𑌵𑍍𑌯𑌾 𑌜𑍁𑌹𑍍𑌵𑌾᳴𑌨 \ul{𑌆}\-𑌸𑌨𑌿᳴।
\-\ul{𑌇}\-𑌮𑌂 𑌨𑍋᳴ \ul{𑌯}\-𑌜𑍍𑌞\-\ul{𑌮}\-𑌮𑍃𑌤𑍇᳴𑌷𑍁 𑌧𑍇𑌹𑌿।
\-\ul{𑌇}\-𑌮𑌾 \ul{𑌹}\-𑌵𑍍𑌯𑌾 𑌜𑌾᳴𑌤𑌵𑍇𑌦𑍋 𑌜𑍁𑌷𑌸𑍍𑌵।
\-\ul{𑌸𑍍𑌤𑍋}\-𑌕𑌾𑌨𑌾᳴𑌮\-\ul{𑌗𑍍𑌨𑍇} 𑌮𑍇𑌦᳴𑌸𑍋 \ul{𑌘𑍃}\-𑌤𑌸𑍍𑌯᳴।
𑌹𑍋\-\ul{𑌤𑌃} 𑌪𑍍𑌰𑌾𑌶𑌾᳴𑌨 𑌪𑍍𑌰\-\ul{𑌥}\-𑌮𑍋 \ul{𑌨𑌿}\-𑌷𑌦𑍍𑌯᳴।
\-\ul{𑌘𑍃}\-𑌤𑌵᳴𑌨𑍍𑌤𑌃 𑌪𑌾𑌵𑌕 𑌤𑍇।
\-\ul{𑌸𑍍𑌤𑍋}\-𑌕𑌾𑌃 𑌶𑍍𑌚𑍋᳴𑌤\-\ul{𑌨𑍍𑌤𑌿} 𑌮𑍇𑌦᳴𑌸𑌃।
𑌸𑍍𑌵𑌧᳴𑌰𑍍𑌮𑌂 \ul{𑌦𑍇}\-𑌵𑌵𑍀᳴𑌤𑌯𑍇॥17॥

%3.6.7.2
𑌶𑍍𑌰𑍇𑌷𑍍𑌠𑌂᳴ 𑌨𑍋 𑌧𑍇\-\ul{𑌹𑌿} 𑌵𑌾𑌰𑍍𑌯𑌮𑍍॑।
𑌤𑍁𑌭𑍍𑌯𑍟᳴ \ul{𑌸𑍍𑌤𑍋}\-𑌕𑌾 𑌘𑍃᳴\-\ul{𑌤}\-𑌶𑍍𑌚𑍁𑌤𑌃᳴।
𑌅\-\ul{𑌗𑍍𑌨𑍇} 𑌵𑌿𑌪𑍍𑌰𑌾᳴𑌯 𑌸𑌨𑍍𑌤𑍍𑌯।
𑌋\-\ul{𑌷𑌿𑌃} 𑌶𑍍𑌰𑍇\-\ul{𑌷𑍍𑌠𑌃} 𑌸𑌮𑌿᳴𑌧𑍍𑌯𑌸𑍇।
\-\ul{𑌯}\-𑌜𑍍𑌞𑌸𑍍𑌯᳴ 𑌪𑍍𑌰𑌾\-\ul{𑌵𑌿}\-𑌤𑌾 𑌭᳴𑌵।
𑌤𑍁𑌭𑍍𑌯𑍟᳴ 𑌶𑍍𑌚𑍋𑌤𑌨𑍍𑌤𑍍𑌯𑌧𑍍𑌰𑌿𑌗𑍋 𑌶𑌚𑍀𑌵𑌃।
\-\ul{𑌸𑍍𑌤𑍋}\-𑌕𑌾𑌸𑍋᳴ 𑌅\-\ul{𑌗𑍍𑌨𑍇} 𑌮𑍇𑌦᳴𑌸𑍋 \ul{𑌘𑍃}\-𑌤𑌸𑍍𑌯᳴।
\-\ul{𑌕}\-\-\ul{𑌵𑌿}\-\-\ul{𑌶}\-𑌸𑍍𑌤𑍋 𑌬𑍃᳴\-\ul{𑌹}\-𑌤𑌾 \ul{𑌭𑌾}\-𑌨𑍁𑌨𑌾𑌗𑌾𑌃॑।
\-\ul{𑌹}\-𑌵𑍍𑌯𑌾 𑌜𑍁᳴𑌷𑌸𑍍𑌵 𑌮𑍇𑌧𑌿𑌰।
𑌓𑌜𑌿᳴𑌷𑍍𑌠𑌨𑍍𑌤𑍇 𑌮\-\ul{𑌧𑍍𑌯}\-𑌤𑍋 𑌮𑍇\-\ul{𑌦} 𑌉𑌦𑍍𑌭𑍃᳴𑌤𑌮𑍍।
𑌪𑍍𑌰 𑌤𑍇᳴ \ul{𑌵}\-𑌯𑌂 𑌦᳴𑌦𑌾𑌮𑌹𑍇।
𑌶𑍍𑌚𑍋𑌤᳴𑌨𑍍𑌤𑌿 𑌤𑍇 𑌵𑌸𑍋 \ul{𑌸𑍍𑌤𑍋}\-𑌕𑌾 𑌅𑌧𑌿᳴\-\ul{𑌤𑍍𑌵}\-𑌚𑌿।
𑌪𑍍𑌰\-\ul{𑌤𑌿} 𑌤𑌾𑌨𑍍𑌦𑍇᳴\-\ul{𑌵}\-𑌶𑍋𑌵𑌿᳴𑌹𑌿॥18॥\anuvakamend[\-\ul{𑌦𑍇}\-𑌵𑌵𑍀᳴𑌤\-\ul{𑌯} 𑌉𑌦𑍍𑌭𑍃᳴\-\ul{𑌤}\-𑌨𑍍𑌤𑍍𑌰𑍀𑌣𑌿᳴ 𑌚]

%3.6.8.1
𑌆𑌵𑍃᳴𑌤𑍍𑌰𑌹𑌣𑌾 𑌵𑍃\-\ul{𑌤𑍍𑌰}\-𑌹\-\ul{𑌭𑌿𑌃} 𑌶𑍁𑌷𑍍𑌮𑍈𑌃॑।
𑌇𑌨𑍍𑌦𑍍𑌰᳴ \ul{𑌯𑌾}\-𑌤𑌨𑍍𑌨𑌮𑍋᳴𑌭𑌿𑌰𑌗𑍍𑌨𑍇 \ul{𑌅}\-𑌰𑍍𑌵𑌾𑌕𑍍।
\-\ul{𑌯𑍁}\-𑌵𑍞 𑌰𑌾𑌧𑍋᳴\-\ul{𑌭𑌿}\-𑌰𑌕᳴𑌵𑍇𑌭𑌿𑌰𑌿𑌨𑍍𑌦𑍍𑌰।
𑌅𑌗𑍍𑌨𑍇᳴ \ul{𑌅}\-𑌸𑍍𑌮𑍇 𑌭᳴𑌵𑌤𑌮𑍁\-\ul{𑌤𑍍𑌤}\-𑌮𑍇𑌭𑌿𑌃᳴।
𑌹𑍋𑌤𑌾᳴ 𑌯𑌕𑍍𑌷𑌦𑌿\-\ul{𑌨𑍍𑌦𑍍𑌰𑌾}\-𑌗𑍍𑌨𑍀।
𑌛𑌾𑌗᳴𑌸𑍍𑌯 \ul{𑌵}\-𑌪𑌾\-\ul{𑌯𑌾} 𑌮𑍇𑌦᳴𑌸𑌃।
\-\ul{𑌜𑍁}\-𑌷𑍇𑌤𑌾𑍞᳴ \ul{𑌹}\-𑌵𑌿𑌃।
𑌹𑍋\-\ul{𑌤}\-𑌰𑍍𑌯𑌜᳴।
𑌵𑌿𑌹𑍍𑌯\-\ul{𑌖𑍍𑌯}\-𑌨𑍍𑌮𑌨᳴\-\ul{𑌸𑌾} 𑌵𑌸𑍍𑌯᳴ \ul{𑌇}\-𑌚𑍍𑌛𑌨𑍍।
𑌇𑌨𑍍𑌦𑍍𑌰𑌾॑𑌗𑍍𑌨𑍀 \ul{𑌜𑍍𑌞𑌾}\-𑌸 \ul{𑌉}\-𑌤 𑌵𑌾᳴ 𑌸\-\ul{𑌜𑌾}\-𑌤𑌾𑌨𑍍॥19॥

%3.6.8.2
𑌨𑌾𑌨𑍍𑌯𑌾 \ul{𑌯𑍁}\-𑌵𑌤𑍍𑌪𑍍𑌰𑌮᳴𑌤𑌿𑌰\-\ul{𑌸𑍍𑌤𑌿} 𑌮𑌹𑍍𑌯𑌮𑍍॑।
𑌸 \ul{𑌵𑌾𑌂} 𑌧𑌿𑌯𑌂᳴ 𑌵𑌾\-\ul{𑌜}\-𑌯𑌨𑍍𑌤𑍀᳴𑌮𑌤𑌕𑍍𑌷𑌮𑍍।
𑌹𑍋𑌤𑌾᳴ 𑌯𑌕𑍍𑌷𑌦𑌿\-\ul{𑌨𑍍𑌦𑍍𑌰𑌾}\-𑌗𑍍𑌨𑍀।
\-\ul{𑌪𑍁}\-\-\ul{𑌰𑍋}\-𑌡𑌾𑌶᳴𑌸𑍍𑌯 \ul{𑌜𑍁}\-𑌷𑍇𑌤𑌾𑍞᳴ \ul{𑌹}\-𑌵𑌿𑌃।
𑌹𑍋\-\ul{𑌤}\-𑌰𑍍𑌯𑌜᳴।
𑌤𑍍𑌵𑌾𑌮𑍀᳴𑌡𑌤𑍇 𑌅\-\ul{𑌜𑌿}\-𑌰𑌂 \ul{𑌦𑍂}\-𑌤𑍍𑌯𑌾᳴𑌯।
\-\ul{𑌹}\-𑌵𑌿𑌷𑍍𑌮᳴\-\ul{𑌨𑍍𑌤𑌃} 𑌸\-\ul{𑌦}\-𑌮𑌿𑌨𑍍𑌮𑌾𑌨𑍁᳴𑌷𑌾𑌸𑌃।
𑌯𑌸𑍍𑌯᳴ \ul{𑌦𑍇}\-𑌵𑍈𑌰𑌾𑌸᳴𑌦𑍋 \ul{𑌬}\-𑌰𑍍‌॒\mbox{}𑌹𑌿𑌰᳴𑌗𑍍𑌨𑍇।
𑌅𑌹𑌾॑𑌨𑍍𑌯𑌸𑍍𑌮𑍈 \ul{𑌸𑍁}\-𑌦𑌿𑌨𑌾᳴ 𑌭𑌵𑌨𑍍𑌤𑍁।
𑌹𑍋𑌤𑌾᳴ 𑌯𑌕𑍍𑌷\-\ul{𑌦}\-𑌗𑍍𑌨𑌿𑌮𑍍।
\-\ul{𑌪𑍁}\-\-\ul{𑌰𑍋}\-𑌡𑌾𑌶᳴𑌸𑍍𑌯 \ul{𑌜𑍁}\-𑌷𑌤𑌾𑍞᳴ \ul{𑌹}\-𑌵𑌿𑌃।
𑌹𑍋\-\ul{𑌤}\-𑌰𑍍𑌯𑌜᳴॥20॥\anuvakamend[\-\ul{𑌸}\-\-\ul{𑌜𑌾}\-𑌤𑌾\-\ul{𑌨}\-𑌗𑍍𑌨𑌿𑌨𑍍𑌦𑍍𑌵𑍇 𑌚᳴]

%3.6.9.1
\-\ul{𑌗𑍀}\-𑌰𑍍𑌭𑌿𑌰𑍍𑌵𑌿\-\ul{𑌪𑍍𑌰𑌃} 𑌪𑍍𑌰𑌮᳴𑌤𑌿\-\ul{𑌮𑌿}\-𑌚𑍍𑌛𑌮𑌾᳴𑌨𑌃।
𑌈𑌟𑍍𑌟𑍇᳴ \ul{𑌰}\-𑌯𑌿𑌂 \ul{𑌯}\-𑌶𑌸𑌂᳴ 𑌪𑍂\-\ul{𑌰𑍍𑌵}\-𑌭𑌾𑌜𑌮𑍍॑।
𑌇𑌨𑍍𑌦𑍍𑌰𑌾॑𑌗𑍍𑌨𑍀 𑌵𑍃𑌤𑍍𑌰𑌹𑌣𑌾 𑌸𑍁𑌵𑌜𑍍𑌰𑌾।
𑌪𑍍𑌰 \ul{𑌣𑍋} 𑌨𑌵𑍍𑌯𑍇᳴𑌭𑌿𑌸𑍍𑌤𑌿𑌰𑌤𑌂 \ul{𑌦𑍇}\-𑌷𑍍𑌣𑍈𑌃।
𑌮𑌾𑌚𑍍𑌛𑍇॑𑌦𑍍𑌮 \ul{𑌰}\-𑌶𑍍𑌮𑍀𑍞𑌰𑌿\-\ul{𑌤𑌿} 𑌨𑌾𑌧᳴𑌮𑌾𑌨𑌾𑌃।
\-\ul{𑌪𑌿}\-\-\ul{𑌤𑍃}\-𑌣𑌾𑍞 𑌶𑌕𑍍𑌤𑍀᳴𑌰\-\ul{𑌨𑍁}\-\-𑌯𑌚𑍍𑌛᳴𑌮𑌾𑌨𑌾𑌃।
\-\ul{𑌇}\-\-\ul{𑌨𑍍𑌦𑍍𑌰𑌾}\-𑌗𑍍𑌨𑌿\-\ul{𑌭𑍍𑌯𑌾𑌂} 𑌕𑌂 𑌵𑍃𑌷᳴𑌣𑍋 𑌮𑌦𑌨𑍍𑌤𑌿।
𑌤𑌾𑌹𑍍𑌯𑌦𑍍𑌰𑍀᳴ \ul{𑌧𑌿}\-𑌷𑌣𑌾᳴𑌯𑌾 \ul{𑌉}\-𑌪𑌸𑍍𑌥𑍇॑।
\-\ul{𑌅}\-𑌗𑍍𑌨𑌿𑍞 𑌸𑍁᳴\-\ul{𑌦𑍀}\-𑌤𑌿𑍞 \ul{𑌸𑍁}\-𑌦𑍃𑌶𑌂᳴ \ul{𑌗𑍃}\-𑌣𑌨𑍍𑌤𑌃᳴।
\-\ul{𑌨}\-\-\ul{𑌮}\-𑌸𑍍𑌯𑌾\-\ul{𑌮}\-𑌸𑍍𑌤𑍍𑌵𑍇𑌡𑍍𑌯𑌂᳴ 𑌜𑌾𑌤𑌵𑍇𑌦𑌃।
𑌤𑍍𑌵𑌾𑌂 \ul{𑌦𑍂}\-𑌤𑌮᳴\-\ul{𑌰}\-𑌤𑌿𑍞 𑌹᳴\-\ul{𑌵𑍍𑌯}\-𑌵𑌾𑌹𑌮𑍍॑।
\-\ul{𑌦𑍇}\-𑌵𑌾 𑌅᳴𑌕𑍃𑌣𑍍𑌵\-\ul{𑌨𑍍𑌨}\-𑌮𑍃𑌤᳴\-\ul{𑌸𑍍𑌯} 𑌨𑌾𑌭𑌿𑌮𑍍॑॥21॥\anuvakamend[\-\ul{𑌜𑌾}\-\-\ul{𑌤}\-\-\ul{𑌵𑍇}\-\-\ul{𑌦𑍋} 𑌦𑍍𑌵𑍇 𑌚᳴]

%3.6.10.1
𑌤𑍍𑌵𑍟 𑌹𑍍𑌯᳴𑌗𑍍𑌨𑍇 𑌪𑍍𑌰\-\ul{𑌥}\-𑌮𑍋 \ul{𑌮}\-𑌨𑍋𑌤𑌾॑।
\-\ul{𑌅}\-𑌸𑍍𑌯𑌾 \ul{𑌧𑌿}\-𑌯𑍋 𑌅𑌭᳴𑌵𑍋 𑌦\-\ul{𑌸𑍍𑌮}\-𑌹𑍋𑌤𑌾॑।
𑌤𑍍𑌵𑍞 𑌸𑍀𑌂 𑌵𑍃𑌷𑌨𑍍𑌨𑌕𑍃𑌣𑍋\-\ul{𑌰𑍍𑌦𑍁}\-𑌷𑍍𑌟𑌰𑍀᳴𑌤𑍁।
𑌸\-\ul{𑌹𑍋} 𑌵𑌿𑌶𑍍𑌵᳴\-\ul{𑌸𑍍𑌮𑍈} 𑌸𑌹᳴\-\ul{𑌸𑍇} 𑌸𑌹᳴𑌧𑍍𑌯𑍈।
𑌅\-\ul{𑌧𑌾} 𑌹𑍋\-\ul{𑌤𑌾} 𑌨𑍍𑌯᳴𑌸𑍀\-\ul{𑌦𑍋} 𑌯𑌜𑍀᳴𑌯𑌾𑌨𑍍।
\-\ul{𑌇}\-𑌡\-\ul{𑌸𑍍𑌪}\-𑌦 \ul{𑌇}\-𑌷\-\ul{𑌯}\-𑌨𑍍𑌨𑍀\-\ul{𑌡𑍍𑌯𑌃} 𑌸𑌨𑍍।
𑌤𑌂 \ul{𑌤𑍍𑌵𑌾} 𑌨𑌰𑌃᳴ 𑌪𑍍𑌰\-\ul{𑌥}\-𑌮𑌂 𑌦𑍇᳴\-\ul{𑌵}\-𑌯𑌨𑍍𑌤𑌃᳴।
\-\ul{𑌮}\-𑌹𑍋 \ul{𑌰𑌾}\-𑌯𑍇 \ul{𑌚𑌿}\-𑌤𑌯᳴\-\ul{𑌨𑍍𑌤𑍋} 𑌅𑌨𑍁᳴𑌗𑍍𑌮𑌨𑍍।
\-\ul{𑌵𑍃}\-𑌤𑍇\-\ul{𑌵} 𑌯𑌨𑍍𑌤𑌂᳴ \ul{𑌬}\-𑌹𑍁𑌭𑌿᳴𑌰𑍍𑌵\-\ul{𑌸}\-𑌵𑍍𑌯𑍈𑌃॑।
𑌤𑍍𑌵𑍇 \ul{𑌰}\-𑌯𑌿𑌂 𑌜𑌾᳴\-\ul{𑌗𑍃}\-𑌵𑌾𑍞\-\ul{𑌸𑍋} 𑌅𑌨𑍁᳴𑌗𑍍𑌮𑌨𑍍॥22॥

%3.6.10.2
𑌰𑍁𑌶᳴𑌨𑍍𑌤\-\ul{𑌮}\-𑌗𑍍𑌨𑌿𑌂 𑌦᳴𑌰𑍍‌\mbox{}\-\ul{𑌶}\-𑌤𑌂 \ul{𑌬𑍃}\-𑌹𑌨𑍍𑌤𑌮𑍍॑।
\-\ul{𑌵}\-𑌪𑌾𑌵᳴𑌨𑍍𑌤𑌂 \ul{𑌵𑌿}\-𑌶𑍍𑌵𑌹𑌾᳴ 𑌦𑍀\-\ul{𑌦𑌿}\-𑌵𑌾𑍞𑌸𑌮𑍍॑।
\-\ul{𑌪}\-𑌦𑌂 \ul{𑌦𑍇}\-𑌵\-\ul{𑌸𑍍𑌯} 𑌨𑌮᳴𑌸𑌾 \ul{𑌵𑌿}\-𑌯𑌨𑍍𑌤𑌃᳴।
\-\ul{𑌶𑍍𑌰}\-\-\ul{𑌵}\-𑌸𑍍𑌯\-\ul{𑌵𑌃} 𑌶𑍍𑌰𑌵᳴ 𑌆\-\ul{𑌪}\-𑌨𑍍𑌨𑌮𑍃᳴𑌕𑍍𑌤𑌮𑍍।
𑌨𑌾𑌮𑌾᳴𑌨𑌿 𑌚𑌿𑌦𑍍𑌦𑌧𑌿𑌰𑍇 \ul{𑌯}\-𑌜𑍍𑌞𑌿𑌯𑌾᳴𑌨𑌿।
\-\ul{𑌭}\-𑌦𑍍𑌰𑌾𑌯𑌾𑌂 𑌤𑍇 𑌰𑌣𑌯\-\ul{𑌨𑍍𑌤} 𑌸𑌨𑍍𑌦𑍃᳴𑌷𑍍𑌟𑍗।
𑌤𑍍𑌵𑌾𑌂 𑌵᳴𑌰𑍍𑌧𑌨𑍍𑌤𑌿 \ul{𑌕𑍍𑌷𑌿}\-𑌤𑌯𑌃᳴ 𑌪𑍃\-\ul{𑌥𑌿}\-𑌵𑍍𑌯𑌾𑌮𑍍।
𑌤𑍍𑌵𑍞 𑌰𑌾𑌯᳴ \ul{𑌉}\-𑌭𑌯𑌾᳴\-\ul{𑌸𑍋} 𑌜𑌨𑌾᳴𑌨𑌾𑌮𑍍।
𑌤𑍍𑌵𑌂 \ul{𑌤𑍍𑌰𑌾}\-𑌤𑌾 𑌤᳴𑌰\-\ul{𑌣𑍇} 𑌚𑍇𑌤𑍍𑌯𑍋᳴𑌽𑌭𑍂𑌃।
\-\ul{𑌪𑌿}\-𑌤𑌾 \ul{𑌮𑌾}\-𑌤𑌾 𑌸\-\ul{𑌦}\-𑌮𑌿𑌨𑍍𑌮𑌾𑌨𑍁᳴𑌷𑌾𑌣𑌾𑌮𑍍॥23॥

%3.6.10.3
𑌸\-\ul{𑌪}\-𑌰𑍍𑌯𑍇\-\ul{𑌣𑍍𑌯𑌃} 𑌸 \ul{𑌪𑍍𑌰𑌿}\-𑌯𑍋 \ul{𑌵𑌿}\-𑌕𑍍𑌷𑍍𑌵᳴𑌗𑍍𑌨𑌿𑌃।
𑌹𑍋𑌤𑌾᳴ \ul{𑌮}\-𑌨𑍍𑌦𑍍𑌰𑍋 𑌨𑌿𑌷᳴𑌸𑌾\-\ul{𑌦𑌾} 𑌯𑌜𑍀᳴𑌯𑌾𑌨𑍍।
𑌤𑌂 𑌤𑍍𑌵𑌾᳴ \ul{𑌵}\-𑌯𑌂 𑌦\-\ul{𑌮} 𑌆 𑌦𑍀᳴\-\ul{𑌦𑌿}\-𑌵𑌾𑍞𑌸𑌮𑍍॑।
𑌉𑌪᳴\-\ul{𑌜𑍍𑌞𑍁}\-𑌬𑌾\-\ul{𑌧𑍋} 𑌨𑌮᳴𑌸𑌾 𑌸𑌦𑍇𑌮।
𑌤𑌂 𑌤𑍍𑌵𑌾᳴ \ul{𑌵}\-𑌯𑍞 \ul{𑌸𑍁}\-𑌧𑌿\-\ul{𑌯𑍋} 𑌨𑌵𑍍𑌯᳴𑌮𑌗𑍍𑌨𑍇।
\-\ul{𑌸𑍁}\-\-\ul{𑌮𑍍𑌨𑌾}\-𑌯𑌵᳴ 𑌈𑌮𑌹𑍇 𑌦𑍇\-\ul{𑌵}\-𑌯𑌨𑍍𑌤𑌃᳴।
𑌤𑍍𑌵𑌂 𑌵𑌿𑌶𑍋᳴ 𑌅𑌨\-\ul{𑌯𑍋} 𑌦𑍀𑌦𑍍𑌯𑌾᳴𑌨𑌃।
\-\ul{𑌦𑌿}\-𑌵𑍋 𑌅᳴𑌗𑍍𑌨𑍇 𑌬𑍃\-\ul{𑌹}\-𑌤𑌾 𑌰𑍋᳴\-\ul{𑌚}\-𑌨𑍇𑌨᳴।
\-\ul{𑌵𑌿}\-𑌶𑌾𑌂 \ul{𑌕}\-𑌵𑌿𑌂 \ul{𑌵𑌿}\-𑌶𑍍𑌪\-\ul{𑌤𑌿}\-\-\ul{𑍞} 𑌶𑌶𑍍𑌵᳴𑌤𑍀𑌨𑌾𑌮𑍍।
\-\ul{𑌨𑌿}\-𑌤𑍋𑌶᳴𑌨𑌂 𑌵𑍃\-\ul{𑌷}\-𑌭𑌂 𑌚᳴𑌰𑍍‌\mbox{}𑌷\-\ul{𑌣𑍀}\-𑌨𑌾𑌮𑍍॥24॥

%3.6.10.4
𑌪𑍍𑌰𑍇𑌤𑍀᳴𑌷𑌣𑌿 \ul{𑌮𑌿}\-𑌷𑌯᳴𑌨𑍍𑌤𑌂 𑌪𑌾\-\ul{𑌵}\-𑌕𑌮𑍍।
𑌰𑌾𑌜᳴𑌨𑍍𑌤\-\ul{𑌮}\-𑌗𑍍𑌨𑌿𑌂 𑌯᳴\-\ul{𑌜}\-𑌤𑍞 𑌰᳴\-\ul{𑌯𑍀}\-𑌣𑌾𑌮𑍍।
𑌸𑍋 𑌅᳴𑌗𑍍𑌨 𑌈𑌜𑍇 𑌶\-\ul{𑌶}\-𑌮𑍇 \ul{𑌚} 𑌮𑌰𑍍𑌤𑌃᳴।
𑌯\-\ul{𑌸𑍍𑌤} 𑌆𑌨᳴\-\ul{𑌟𑍍𑌥𑍍𑌸}\-𑌮𑌿𑌧𑌾᳴ \ul{𑌹}\-𑌵𑍍𑌯𑌦𑌾᳴𑌤𑌿𑌮𑍍।
𑌯 𑌆𑌹𑍁᳴\-\ul{𑌤𑌿𑌂} 𑌪\-\ul{𑌰𑌿} 𑌵𑍇\-\ul{𑌦𑌾} 𑌨𑌮𑍋᳴𑌭𑌿𑌃।
𑌵𑌿𑌶𑍍𑌵𑍇𑌥𑍍𑌸\-\ul{𑌵𑌾}\-𑌮𑌾 𑌦᳴𑌧\-\ul{𑌤𑍇} 𑌤𑍍𑌵𑍋𑌤𑌃᳴।
\-\ul{𑌅}\-𑌸𑍍𑌮𑌾 𑌉᳴ \ul{𑌤𑍇} 𑌮𑌹𑌿᳴ \ul{𑌮}\-𑌹𑍇 𑌵𑌿᳴𑌧𑍇𑌮।
𑌨𑌮𑍋᳴𑌭𑌿𑌰𑌗𑍍𑌨𑍇 \ul{𑌸}\-𑌮𑌿\-\ul{𑌧𑍋}\-𑌤 \ul{𑌹}\-𑌵𑍍𑌯𑍈𑌃।
𑌵𑍇𑌦𑍀᳴𑌸𑍂𑌨𑍋 𑌸𑌹𑌸𑍋 \ul{𑌗𑍀}\-𑌰𑍍𑌭𑌿\-\ul{𑌰𑍁}\-𑌕𑍍𑌥𑍈𑌃।
𑌆 𑌤𑍇᳴ \ul{𑌭}\-𑌦𑍍𑌰𑌾𑌯𑌾𑍞᳴ 𑌸𑍁\-\ul{𑌮}\-𑌤𑍗 𑌯᳴𑌤𑍇𑌮॥25॥

%3.6.10.5
𑌆 𑌯\-\ul{𑌸𑍍𑌤}\-𑌤\-\ul{𑌨𑍍𑌥} 𑌰𑍋𑌦᳴\-\ul{𑌸𑍀} 𑌵𑌿\-\ul{𑌭𑌾}\-𑌸𑌾।
𑌶𑍍𑌰𑌵𑍋᳴𑌭𑌿𑌶𑍍𑌚 𑌶𑍍𑌰\-\ul{𑌵}\-𑌸𑍍𑌯᳴𑌸𑍍𑌤𑌰𑍁᳴𑌤𑍍𑌰𑌃।
\-\ul{𑌬𑍃}\-𑌹\-\ul{𑌦𑍍𑌭𑌿}\-𑌰𑍍𑌵𑌾\-\ul{𑌜𑍈𑌃} 𑌸𑍍𑌥𑌵𑌿᳴𑌰𑍇𑌭𑌿\-\ul{𑌰}\-𑌸𑍍𑌮𑍇।
\-\ul{𑌰𑍇}\-𑌵𑌦𑍍𑌭𑌿᳴𑌰𑌗𑍍𑌨𑍇 𑌵𑌿\-\ul{𑌤}\-𑌰𑌂 𑌵𑌿 𑌭𑌾᳴𑌹𑌿।
\-\ul{𑌨𑍃}\-𑌵𑌦𑍍𑌵᳴\-\ul{𑌸𑍋} 𑌸\-\ul{𑌦}\-𑌮𑌿𑌦𑍍𑌧𑍇॑\-\ul{𑌹𑍍𑌯}\-𑌸𑍍𑌮𑍇।
𑌭𑍂𑌰𑌿᳴\-\ul{𑌤𑍋}\-𑌕𑌾\-\ul{𑌯} 𑌤𑌨᳴𑌯𑌾𑌯 \ul{𑌪}\-𑌶𑍍𑌵𑌃।
\-\ul{𑌪𑍂}\-𑌰𑍍𑌵𑍀𑌰𑌿𑌷𑍋᳴ 𑌬𑍃\-\ul{𑌹}\-𑌤𑍀\-\ul{𑌰𑌾}\-𑌰𑍇 𑌅᳴𑌘𑌾𑌃।
\-\ul{𑌅}\-𑌸𑍍𑌮𑍇 \ul{𑌭}\-𑌦𑍍𑌰𑌾 𑌸𑍗॑𑌶𑍍𑌰\-\ul{𑌵}\-𑌸𑌾𑌨𑌿᳴ 𑌸𑌨𑍍𑌤𑍁।
\-\ul{𑌪𑍁}\-𑌰𑍂𑌣𑍍𑌯᳴𑌗𑍍𑌨𑍇 𑌪𑍁\-\ul{𑌰𑍁}\-𑌧𑌾 \ul{𑌤𑍍𑌵𑌾}\-𑌯𑌾।
𑌵𑌸𑍂᳴𑌨𑌿 𑌰𑌾𑌜\-\ul{𑌨𑍍𑌵}\-𑌸𑍁𑌤𑌾᳴𑌤𑍇 𑌅𑌶𑍍𑌯𑌾𑌮𑍍।
\-\ul{𑌪𑍁}\-𑌰𑍂\-\ul{𑌣𑌿} 𑌹𑌿 𑌤𑍍𑌵𑍇 𑌪𑍁᳴𑌰𑍁𑌵𑌾\-\ul{𑌰} 𑌸𑌨𑍍𑌤𑌿᳴।
𑌅\-\ul{𑌗𑍍𑌨𑍇} 𑌵𑌸𑍁᳴ 𑌵𑌿\-\ul{𑌧}\-𑌤𑍇 𑌰𑌾𑌜᳴\-\ul{𑌨𑌿}\-𑌤𑍍𑌵𑍇॥26॥\anuvakamend[\-\ul{𑌜𑌾}\-\-\ul{𑌗𑍃}\-𑌵𑌾𑍞\-\ul{𑌸𑍋} 𑌅𑌨𑍁᳴\-\ul{𑌗𑍍𑌮}\-𑌨𑍍𑌮𑌾𑌨𑍁᳴𑌷𑌾𑌣𑌾𑌞𑍍𑌚𑌰𑍍‌\mbox{}𑌷\-\ul{𑌣𑍀}\-𑌨𑌾𑌂 𑌯᳴𑌤𑍇𑌮𑌾\-\ul{𑌶𑍍𑌯𑌾}\-𑌨𑍍𑌦𑍍𑌵𑍇 𑌚᳴]

%3.6.11.1
𑌆𑌭᳴𑌰𑌤𑍞 𑌶𑌿𑌕𑍍𑌷𑌤𑌂 𑌵𑌜𑍍𑌰𑌬𑌾𑌹𑍂।
\-\ul{𑌅}\-𑌸𑍍𑌮𑌾𑍞 𑌇᳴𑌨𑍍𑌦𑍍𑌰𑌾𑌗𑍍𑌨𑍀 𑌅𑌵\-\ul{𑌤}\-\-\ul{𑍞} 𑌶𑌚𑍀᳴𑌭𑌿𑌃।
\-\ul{𑌇}\-𑌮𑍇 𑌨𑍁 𑌤𑍇 \ul{𑌰}\-𑌶𑍍𑌮\-\ul{𑌯𑌃} 𑌸𑍂𑌰𑍍𑌯᳴𑌸𑍍𑌯।
𑌯𑍇𑌭𑌿𑌃᳴ 𑌸\-\ul{𑌪𑌿}\-𑌤𑍍𑌵𑌂 \ul{𑌪𑌿}\-𑌤𑌰𑍋᳴ \ul{𑌨} 𑌆𑌯𑌨𑍍।
𑌹𑍋𑌤𑌾᳴ 𑌯𑌕𑍍𑌷𑌦𑌿\-\ul{𑌨𑍍𑌦𑍍𑌰𑌾}\-𑌗𑍍𑌨𑍀।
𑌛𑌾𑌗᳴𑌸𑍍𑌯 \ul{𑌹}\-𑌵𑌿\-\ul{𑌷} 𑌆𑌤𑍍𑌤𑌾᳴\-\ul{𑌮}\-𑌦𑍍𑌯।
\-\ul{𑌮}\-\-\ul{𑌧𑍍𑌯}\-𑌤𑍋 𑌮𑍇\-\ul{𑌦} 𑌉𑌦𑍍𑌭𑍃᳴𑌤𑌮𑍍।
\-\ul{𑌪𑍁}\-𑌰𑌾 𑌦𑍍𑌵𑍇𑌷𑍋॑𑌭𑍍𑌯𑌃।
\-\ul{𑌪𑍁}\-𑌰𑌾 𑌪𑍗𑌰𑍁᳴𑌷𑍇𑌯𑍍𑌯𑌾 \ul{𑌗𑍃}\-𑌭𑌃।
𑌘𑌸𑍍𑌤𑌾॑\-\ul{𑌨𑍍𑌨𑍂}\-𑌨𑌮𑍍॥27॥

%3.6.11.2
\-\ul{𑌘𑌾}\-𑌸𑍇 𑌅᳴𑌜𑍍𑌰𑌾\-\ul{𑌣𑌾𑌂} 𑌯𑌵᳴𑌸𑌪𑍍𑌰𑌥𑌮𑌾𑌨𑌾𑌮𑍍।
\-\ul{𑌸𑍁}\-𑌮𑌤𑍍𑌕𑍍𑌷᳴𑌰𑌾𑌣𑌾𑍞 \ul{𑌶}\-𑌤𑌰𑍁᳴𑌦𑍍𑌰𑌿\-𑌯𑌾𑌣𑌾𑌮𑍍।
\-\ul{𑌅}\-\-\ul{𑌗𑍍𑌨𑌿}\-\-\ul{𑌷𑍍𑌵𑌾}\-𑌤𑍍𑌤𑌾\-\ul{𑌨𑌾𑌂} 𑌪𑍀𑌵𑍋᳴𑌪𑌵𑌸𑌨𑌾𑌨𑌾𑌮𑍍।
\-\ul{𑌪𑌾}\-\-\ul{𑌰𑍍𑌶𑍍𑌵}\-𑌤𑌃 𑌶𑍍𑌰𑍋᳴\-\ul{𑌣𑌿}\-𑌤𑌃 𑌶𑌿᳴𑌤𑌾\-\ul{𑌮}\-𑌤 𑌉᳴𑌥𑍍𑌸𑌾\-\ul{𑌦}\-𑌤𑌃।
𑌅𑌙𑍍𑌗𑌾᳴𑌦\-\ul{𑌙𑍍𑌗𑌾}\-𑌦𑌵᳴𑌤𑍍𑌤𑌾𑌨𑌾𑌮𑍍।
𑌕𑌰᳴𑌤 \ul{𑌏}\-𑌵𑍇\-\ul{𑌨𑍍𑌦𑍍𑌰𑌾}\-𑌗𑍍𑌨𑍀।
\-\ul{𑌜𑍁}\-𑌷𑍇𑌤𑌾𑍞᳴ \ul{𑌹}\-𑌵𑌿𑌃।
𑌹𑍋\-\ul{𑌤}\-𑌰𑍍𑌯𑌜᳴।
\-\ul{𑌦𑍇}\-𑌵𑍇𑌭𑍍𑌯𑍋᳴ 𑌵𑌨𑌸𑍍𑌪𑌤𑍇 \ul{𑌹}\-𑌵𑍀𑍞𑌷𑌿᳴।
𑌹𑌿𑌰᳴𑌣𑍍𑌯𑌪𑌰𑍍𑌣 \ul{𑌪𑍍𑌰}\-𑌦𑌿𑌵᳴\-\ul{𑌸𑍍𑌤𑍇} 𑌅𑌰𑍍𑌥𑌮𑍍॑॥28॥

%3.6.11.3
\-\ul{𑌪𑍍𑌰}\-\-\ul{𑌦}\-\-\ul{𑌕𑍍𑌷𑌿}\-𑌣𑌿𑌦𑍍𑌰᳴\-\ul{𑌶}\-𑌨𑌯𑌾᳴ \ul{𑌨𑌿}\-𑌯𑍂𑌯᳴।
\-\ul{𑌋}\-𑌤𑌸𑍍𑌯᳴ 𑌵𑌕𑍍𑌷𑌿 \ul{𑌪}\-𑌥𑌿\-\ul{𑌭𑍀} 𑌰𑌜𑌿᳴𑌷𑍍𑌠𑍈𑌃।
𑌹𑍋𑌤𑌾᳴ 𑌯\-\ul{𑌕𑍍𑌷}\-𑌦𑍍𑌵\-\ul{𑌨}\-𑌸𑍍𑌪𑌤𑌿᳴\-\ul{𑌮}\-𑌭𑌿𑌹𑌿।
\-\ul{𑌪𑌿}\-𑌷𑍍𑌟𑌤᳴𑌮\-\ul{𑌯𑌾} 𑌰𑌭𑌿᳴𑌷𑍍𑌠𑌯𑌾 𑌰\-\ul{𑌶}\-𑌨𑌯𑌾𑌧𑌿᳴𑌤।
𑌯𑌤𑍍𑌰𑍇॑𑌨𑍍𑌦𑍍𑌰𑌾\-\ul{𑌗𑍍𑌨𑌿}\-𑌯𑍋𑌶𑍍𑌛𑌾𑌗᳴𑌸𑍍𑌯 \ul{𑌹}\-𑌵𑌿𑌷𑌃᳴ \ul{𑌪𑍍𑌰𑌿}\-𑌯𑌾 𑌧𑌾𑌮𑌾᳴𑌨𑌿।
𑌯\-\ul{𑌤𑍍𑌰} 𑌵\-\ul{𑌨}\-𑌸𑍍𑌪𑌤𑍇॑: \ul{𑌪𑍍𑌰𑌿}\-𑌯𑌾 𑌪𑌾𑌥𑌾𑍞᳴𑌸𑌿।
𑌯𑌤𑍍𑌰᳴ \ul{𑌦𑍇}\-𑌵𑌾𑌨𑌾᳴𑌮𑌾\-\ul{𑌜𑍍𑌯}\-𑌪𑌾𑌨𑌾𑌂॑ \ul{𑌪𑍍𑌰𑌿}\-𑌯𑌾 𑌧𑌾𑌮𑌾᳴𑌨𑌿।
𑌯\-\ul{𑌤𑍍𑌰𑌾}\-𑌗𑍍𑌨𑍇𑌰𑍍‌\mbox{}𑌹𑍋𑌤𑍁𑌃᳴ \ul{𑌪𑍍𑌰𑌿}\-𑌯𑌾 𑌧𑌾𑌮𑌾᳴𑌨𑌿।
𑌤\-\ul{𑌤𑍍𑌰𑍈}\-𑌤𑌂 \ul{𑌪𑍍𑌰}\-𑌸𑍍𑌤𑍁𑌤𑍍𑌯𑍇᳴𑌵𑍋\-\ul{𑌪}\-𑌸𑍍𑌤𑍁𑌤𑍍𑌯𑍇᳴ \ul{𑌵𑍋}\-𑌪𑌾𑌵᳴𑌸𑍍𑌰𑌕𑍍𑌷𑌤𑍍।
𑌰𑌭𑍀᳴𑌯𑌾𑍞𑌸𑌮𑌿𑌵 \ul{𑌕𑍃}\-𑌤𑍍𑌵𑍀॥29॥

%3.6.11.4
𑌕𑌰᳴\-\ul{𑌦𑍇}\-𑌵𑌂 \ul{𑌦𑍇}\-𑌵𑍋 𑌵\-\ul{𑌨}\-𑌸𑍍𑌪𑌤𑌿𑌃᳴।
\-\ul{𑌜𑍁}\-𑌷𑌤𑌾𑍞᳴ \ul{𑌹}\-𑌵𑌿𑌃।
𑌹𑍋\-\ul{𑌤}\-𑌰𑍍𑌯𑌜᳴।
\-\ul{𑌪𑌿}\-\-\ul{𑌪𑍍𑌰𑍀}\-𑌹𑌿 \ul{𑌦𑍇}\-𑌵𑌾𑍞 𑌉᳴\-\ul{𑌶}\-𑌤𑍋 𑌯᳴𑌵𑌿𑌷𑍍𑌠।
\-\ul{𑌵𑌿}\-𑌦𑍍𑌵𑌾𑍞 \ul{𑌋}\-𑌤𑍂𑍞𑌰𑍍\mbox{}𑌋᳴𑌤𑍁𑌪𑌤𑍇 𑌯\-\ul{𑌜𑍇}\-𑌹।
𑌯𑍇 𑌦𑍈𑌵𑍍𑌯𑌾᳴ \ul{𑌋}\-𑌤𑍍𑌵𑌿\-\ul{𑌜}\-𑌸𑍍𑌤𑍇𑌭𑌿᳴𑌰𑌗𑍍𑌨𑍇।
𑌤𑍍𑌵𑍞 𑌹𑍋𑌤𑍄᳴𑌣𑌾\-\ul{𑌮}\-𑌸𑍍𑌯𑌾𑌯᳴𑌜𑌿𑌷𑍍𑌠𑌃।
𑌹𑍋𑌤𑌾᳴ 𑌯𑌕𑍍𑌷\-\ul{𑌦}\-𑌗𑍍𑌨𑌿𑍟 𑌸𑍍𑌵𑌿᳴\-\ul{𑌷𑍍𑌟}\-𑌕𑍃𑌤𑌮𑍍॑।
𑌅𑌯𑌾᳴\-\ul{𑌡}\-𑌗𑍍𑌨𑌿𑌰𑌿᳴𑌨𑍍𑌦𑍍𑌰𑌾\-\ul{𑌗𑍍𑌨𑌿}\-𑌯𑍋𑌶𑍍𑌛𑌾𑌗᳴𑌸𑍍𑌯 \ul{𑌹}\-𑌵𑌿𑌷𑌃᳴ \ul{𑌪𑍍𑌰𑌿}\-𑌯𑌾 𑌧𑌾𑌮𑌾᳴𑌨𑌿।
𑌅\-\ul{𑌯𑌾}\-𑌡𑍍𑌵\-\ul{𑌨}\-𑌸𑍍𑌪𑌤𑍇॑: \ul{𑌪𑍍𑌰𑌿}\-𑌯𑌾 𑌪𑌾𑌥𑌾𑍞᳴𑌸𑌿।
𑌅𑌯𑌾॑\-\ul{𑌡𑍍𑌦𑍇}\-𑌵𑌾𑌨𑌾᳴𑌮𑌾\-\ul{𑌜𑍍𑌯}\-𑌪𑌾𑌨𑌾𑌂॑ \ul{𑌪𑍍𑌰𑌿}\-𑌯𑌾 𑌧𑌾𑌮𑌾᳴𑌨𑌿।
𑌯𑌕𑍍𑌷᳴\-\ul{𑌦}\-𑌗𑍍𑌨𑍇𑌰𑍍‌\mbox{}𑌹𑍋𑌤𑍁𑌃᳴ \ul{𑌪𑍍𑌰𑌿}\-𑌯𑌾 𑌧𑌾𑌮𑌾᳴𑌨𑌿।
𑌯\-\ul{𑌕𑍍𑌷}\-𑌥𑍍𑌸𑍍𑌵𑌂 𑌮᳴\-\ul{𑌹𑌿}\-𑌮𑌾𑌨𑌮𑍍॑।
𑌆𑌯᳴𑌜\-\ul{𑌤𑌾}\-𑌮𑍇\-\ul{𑌜𑍍𑌯𑌾} 𑌇𑌷𑌃᳴।
\-\ul{𑌕𑍃}\-𑌣𑍋\-\ul{𑌤𑍁} 𑌸𑍋 𑌅᳴\-\ul{𑌧𑍍𑌵}\-𑌰𑌾 \ul{𑌜𑌾}\-𑌤𑌵𑍇᳴𑌦𑌾𑌃।
\-\ul{𑌜𑍁}\-𑌷𑌤𑌾𑍞᳴ \ul{𑌹}\-𑌵𑌿𑌃।
𑌹𑍋\-\ul{𑌤}\-𑌰𑍍𑌯𑌜᳴॥30॥\anuvakamend[\-\ul{𑌨𑍂}\-𑌨𑌮𑌰𑍍𑌥𑌂᳴ \ul{𑌕𑍃}\-𑌤𑍍𑌵𑍀 𑌪𑌾𑌥𑌾𑍞᳴𑌸𑌿 \ul{𑌸}\-𑌪𑍍𑌤 𑌚᳴]

%3.6.12.1
𑌉𑌪𑍋᳴ \ul{𑌹} 𑌯\-\ul{𑌦𑍍𑌵𑌿}\-𑌦𑌥𑌂᳴ \ul{𑌵𑌾}\-𑌜𑌿\-\ul{𑌨𑍋} 𑌗𑍂𑌃।
\-\ul{𑌗𑍀}\-𑌰𑍍𑌭𑌿𑌰𑍍𑌵𑌿\-\ul{𑌪𑍍𑌰𑌾𑌃} 𑌪𑍍𑌰𑌮᳴𑌤𑌿\-\ul{𑌮𑌿}\-𑌚𑍍𑌛𑌮𑌾᳴𑌨𑌾𑌃।
\-\ul{𑌅}\-𑌰𑍍𑌵\-\ul{𑌨𑍍𑌤𑍋} 𑌨 𑌕𑌾\-\ul{𑌷𑍍𑌠𑌾}\-𑌨𑍍𑌨𑌕𑍍𑌷᳴𑌮𑌾𑌣𑌾𑌃।
\-\ul{𑌇}\-\-\ul{𑌨𑍍𑌦𑍍𑌰𑌾}\-𑌗𑍍𑌨𑍀 𑌜𑍋𑌹𑍁᳴𑌵\-\ul{𑌤𑍋} 𑌨\-\ul{𑌰}\-𑌸𑍍𑌤𑍇।
𑌵𑌨᳴𑌸𑍍𑌪𑌤𑍇 𑌰\-\ul{𑌶}\-𑌨𑌯𑌾᳴\-𑌽\-\ul{𑌭𑌿}\-𑌧𑌾𑌯᳴।
\-\ul{𑌪𑌿}\-𑌷𑍍𑌟𑌤᳴𑌮𑌯𑌾 \ul{𑌵}\-𑌯𑍁𑌨𑌾᳴𑌨𑌿 \ul{𑌵𑌿}\-𑌦𑍍𑌵𑌾𑌨𑍍।
𑌵𑌹᳴ 𑌦𑍇\-\ul{𑌵}\-𑌤𑍍𑌰𑌾 𑌦𑌿᳴𑌧𑌿𑌷𑍋 \ul{𑌹}\-𑌵𑍀𑍞𑌷𑌿᳴।
𑌪𑍍𑌰 𑌚᳴\-\ul{𑌦𑌾}\-𑌤𑌾𑌰᳴\-\ul{𑌮}\-𑌮𑍃𑌤𑍇᳴𑌷𑍁 𑌵𑍋𑌚𑌃।
\-\ul{𑌅}\-𑌗𑍍𑌨𑌿𑍟 𑌸𑍍𑌵𑌿᳴\-\ul{𑌷𑍍𑌟}\-𑌕𑍃𑌤𑌮𑍍॑।
𑌅𑌯𑌾᳴\-\ul{𑌡}\-𑌗𑍍𑌨𑌿𑌰𑌿᳴𑌨𑍍𑌦𑍍𑌰𑌾\-\ul{𑌗𑍍𑌨𑌿}\-𑌯𑍋𑌶𑍍𑌛𑌾𑌗᳴𑌸𑍍𑌯 \ul{𑌹}\-𑌵𑌿𑌷𑌃᳴ \ul{𑌪𑍍𑌰𑌿}\-𑌯𑌾 𑌧𑌾𑌮𑌾᳴𑌨𑌿॥31॥

%3.6.12.2
𑌅\-\ul{𑌯𑌾}\-𑌡𑍍𑌵\-\ul{𑌨}\-𑌸𑍍𑌪𑌤𑍇॑: \ul{𑌪𑍍𑌰𑌿}\-𑌯𑌾 𑌪𑌾𑌥𑌾𑍞᳴𑌸𑌿।
𑌅𑌯𑌾॑\-\ul{𑌡𑍍𑌦𑍇}\-𑌵𑌾𑌨𑌾᳴𑌮𑌾\-\ul{𑌜𑍍𑌯}\-𑌪𑌾𑌨𑌾𑌂॑ \ul{𑌪𑍍𑌰𑌿}\-𑌯𑌾 𑌧𑌾𑌮𑌾᳴𑌨𑌿।
𑌯𑌕𑍍𑌷᳴\-\ul{𑌦}\-𑌗𑍍𑌨𑍇𑌰𑍍‌\mbox{}𑌹𑍋𑌤𑍁𑌃᳴ \ul{𑌪𑍍𑌰𑌿}\-𑌯𑌾 𑌧𑌾𑌮𑌾᳴𑌨𑌿।
𑌯\-\ul{𑌕𑍍𑌷}\-𑌥𑍍𑌸𑍍𑌵𑌂 𑌮᳴\-\ul{𑌹𑌿}\-𑌮𑌾𑌨𑌮𑍍॑।
𑌆𑌯᳴𑌜\-\ul{𑌤𑌾}\-𑌮𑍇\-\ul{𑌜𑍍𑌯𑌾} 𑌇𑌷𑌃᳴।
\-\ul{𑌕𑍃}\-𑌣𑍋\-\ul{𑌤𑍁} 𑌸𑍋 𑌅᳴\-\ul{𑌧𑍍𑌵}\-𑌰𑌾 \ul{𑌜𑌾}\-𑌤𑌵𑍇᳴𑌦𑌾𑌃।
\-\ul{𑌜𑍁}\-𑌷𑌤𑌾𑍞᳴ \ul{𑌹}\-𑌵𑌿𑌃।
𑌅\-\ul{𑌗𑍍𑌨𑍇} 𑌯\-\ul{𑌦}\-𑌦𑍍𑌯 \ul{𑌵𑌿}\-𑌶𑍋 𑌅᳴𑌧𑍍𑌵𑌰𑌸𑍍𑌯 𑌹𑍋𑌤𑌃।
𑌪𑌾𑌵᳴𑌕 𑌶𑍋\-\ul{𑌚𑍇} 𑌵𑍇𑌷𑍍𑌟𑍍𑌵𑍞 𑌹𑌿 𑌯𑌜𑍍𑌵𑌾॑।
\-\ul{𑌋}\-𑌤𑌾 𑌯᳴𑌜𑌾𑌸𑌿 𑌮\-\ul{𑌹𑌿}\-𑌨𑌾 𑌵𑌿𑌯𑌦𑍍𑌭𑍂𑌃।
\-\ul{𑌹}\-𑌵𑍍𑌯𑌾 𑌵᳴𑌹 𑌯𑌵𑌿\-\ul{𑌷𑍍𑌠} 𑌯𑌾 𑌤𑍇᳴ \ul{𑌅}\-𑌦𑍍𑌯॥32॥\anuvakamend[𑌧𑌾𑌮𑌾᳴\-\ul{𑌨𑌿} 𑌭𑍂𑌰𑍇𑌕𑌂᳴ 𑌚]

%3.6.13.1
\-\ul{𑌦𑍇}\-𑌵𑌂 \ul{𑌬}\-\-\ul{𑌰𑍍}\-𑌹𑌿𑌃 𑌸𑍁᳴\-\ul{𑌦𑍇}\-𑌵𑌂 \ul{𑌦𑍇}\-𑌵𑍈𑌃 𑌸𑍍𑌯𑌾\-\ul{𑌥𑍍𑌸𑍁}\-𑌵𑍀𑌰𑌂᳴ \ul{𑌵𑍀}\-𑌰𑍈𑌰𑍍𑌵𑌸𑍍𑌤𑍋॑\-\ul{𑌰𑍍𑌵𑍃}\-𑌜𑍍𑌯𑍇\-\ul{𑌤𑌾}\-𑌕𑍍𑌤𑍋𑌃 𑌪𑍍𑌰𑌭𑍍𑌰𑌿᳴\-\ul{𑌯𑍇}\-𑌤𑌾\-\ul{𑌤𑍍𑌯}\-𑌨𑍍𑌯𑌾\-\ul{𑌨𑍍𑌰𑌾}\-𑌯𑌾 \ul{𑌬}\-\-\ul{𑌰𑍍}\-𑌹𑌿𑌷𑍍𑌮᳴𑌤𑍋 𑌮𑌦𑍇𑌮 𑌵\-\ul{𑌸𑍁}\-𑌵𑌨𑍇᳴ 𑌵\-\ul{𑌸𑍁}\-𑌧𑍇𑌯᳴𑌸𑍍𑌯 𑌵𑍇\-\ul{𑌤𑍁} 𑌯𑌜᳴।
\-\ul{𑌦𑍇}\-𑌵𑍀𑌰𑍍𑌦𑍍𑌵𑌾𑌰𑌃᳴ 𑌸\-\ul{𑌙𑍍𑌘𑌾}\-𑌤𑍇 \ul{𑌵𑌿}\-𑌡𑍍𑌵𑍀𑌰𑍍𑌯𑌾𑌮᳴𑌞𑍍𑌛𑌿\-\ul{𑌥𑌿}\-𑌰𑌾 \ul{𑌧𑍍𑌰𑍁}\-𑌵𑌾 \ul{𑌦𑍇}\-𑌵𑌹𑍂᳴𑌤𑍗 \ul{𑌵}\-𑌥𑍍𑌸 𑌈᳴𑌮𑍇\-\ul{𑌨𑌾}\-𑌸𑍍𑌤𑌰𑍁᳴\-\ul{𑌣} 𑌆𑌮𑌿᳴𑌮𑍀𑌯𑌾𑌤𑍍𑌕𑍁\-\ul{𑌮𑌾}\-𑌰𑍋 \ul{𑌵𑌾} 𑌨𑌵᳴𑌜𑌾\-\ul{𑌤𑍋} 𑌮𑍈\-\ul{𑌨𑌾} 𑌅𑌰𑍍𑌵𑌾᳴ \ul{𑌰𑍇}\-𑌣𑍁𑌕᳴𑌕𑌾\-\ul{𑌟𑌃} 𑌪𑍃𑌣᳴𑌗𑍍𑌵\-\ul{𑌸𑍁}\-𑌵𑌨𑍇᳴ 𑌵\-\ul{𑌸𑍁}\-𑌧𑍇𑌯᳴𑌸𑍍𑌯 𑌵𑌿𑌯\-\ul{𑌨𑍍𑌤𑍁} 𑌯𑌜᳴।
\-\ul{𑌦𑍇}\-𑌵𑍀 \ul{𑌉}\-𑌷𑌾\-\ul{𑌸𑌾}\-𑌨𑌕𑍍𑌤𑌾\-𑌽\-\ul{𑌦𑍍𑌯𑌾}\-𑌸𑍍𑌮𑌿𑌨𑍍‌ \ul{𑌯}\-𑌜𑍍𑌞𑍇 𑌪𑍍𑌰᳴\-\ul{𑌯}\-𑌤𑍍𑌯᳴𑌹𑍍𑌵𑍇\-\ul{𑌤𑌾}\-𑌮𑌪𑌿᳴ \ul{𑌨𑍂}\-𑌨𑌂 𑌦𑍈\-\ul{𑌵𑍀}\-𑌰𑍍𑌵𑌿\-\ul{𑌶𑌃} 𑌪𑍍𑌰𑌾𑌯𑌾᳴𑌸𑌿\-\ul{𑌷𑍍𑌟𑌾}\-\-\ul{𑍞} 𑌸𑍁𑌪𑍍𑌰𑍀᳴\-\ul{𑌤𑍇} 𑌸𑍁𑌧𑌿᳴𑌤𑍇 𑌵\-\ul{𑌸𑍁}\-𑌵𑌨𑍇᳴ 𑌵\-\ul{𑌸𑍁}\-𑌧𑍇𑌯᳴𑌸𑍍𑌯 𑌵𑍀\-\ul{𑌤𑌾𑌂} 𑌯𑌜᳴।
\-\ul{𑌦𑍇}\-𑌵𑍀 𑌜𑍋\-\ul{𑌷𑍍𑌟𑍍𑌰𑍀} 𑌵𑌸𑍁᳴𑌧𑌿\-\ul{𑌤𑍀} 𑌯𑌯𑍋᳴\-\ul{𑌰}\-𑌨𑍍𑌯𑌾\-𑌽𑌘𑌾𑌦𑍍𑌦𑍍𑌵𑍇𑌷𑌾𑍞᳴𑌸𑌿 \ul{𑌯𑍁}\-𑌯\-\ul{𑌵}\-𑌦𑌾𑌨𑍍𑌯𑌾𑌵᳴\-\ul{𑌕𑍍𑌷}\-𑌦𑍍𑌵\-\ul{𑌸𑍁} 𑌵𑌾𑌰𑍍𑌯𑌾᳴\-\ul{𑌣𑌿} 𑌯𑌜᳴𑌮𑌾𑌨𑌾𑌯 𑌵\-\ul{𑌸𑍁}\-𑌵𑌨𑍇᳴ 𑌵\-\ul{𑌸𑍁}\-𑌧𑍇𑌯᳴𑌸𑍍𑌯 𑌵𑍀\-\ul{𑌤𑌾𑌂} 𑌯𑌜᳴।
\-\ul{𑌦𑍇}\-𑌵𑍀 \ul{𑌊}\-𑌰𑍍𑌜𑌾𑌹𑍁᳴\-\ul{𑌤𑍀} 𑌇\-\ul{𑌷}\-𑌮𑍂𑌰𑍍𑌜᳴\-\ul{𑌮}\-𑌨𑍍𑌯𑌾𑌵᳴\-\ul{𑌕𑍍𑌷}\-𑌥𑍍𑌸\-\ul{𑌗𑍍𑌧𑌿}\-\-\ul{𑍞} 𑌸𑌪𑍀᳴𑌤𑌿\-\ul{𑌮}\-𑌨𑍍𑌯𑌾 𑌨𑌵𑍇᳴\-\ul{𑌨} 𑌪𑍂\-\ul{𑌰𑍍𑌵}\-𑌨𑍍𑌦𑌯᳴𑌮𑌾\-\ul{𑌨𑌾𑌃} 𑌸𑍍𑌯𑌾𑌮᳴ 𑌪𑍁\-\ul{𑌰𑌾}\-𑌣𑍇\-\ul{𑌨} 𑌨\-\ul{𑌵}\-𑌨𑍍𑌤𑌾𑌮𑍂𑌰𑍍𑌜᳴\-\ul{𑌮𑍂}\-𑌰𑍍𑌜𑌾𑌹𑍁᳴𑌤𑍀 \ul{𑌊}\-𑌰𑍍𑌜𑌯᳴𑌮𑌾𑌨𑍇 𑌅𑌧𑌾𑌤𑌾𑌂 𑌵\-\ul{𑌸𑍁}\-𑌵𑌨𑍇᳴ 𑌵\-\ul{𑌸𑍁}\-𑌧𑍇𑌯᳴𑌸𑍍𑌯 𑌵𑍀\-\ul{𑌤𑌾𑌂} 𑌯𑌜᳴।
\-\ul{𑌦𑍇}\-𑌵𑌾 𑌦𑍈\-\ul{𑌵𑍍𑌯𑌾} 𑌹𑍋𑌤𑌾᳴\-\ul{𑌰𑌾} 𑌨𑍇𑌷𑍍𑌟𑌾᳴\-\ul{𑌰𑌾} 𑌪𑍋𑌤𑌾᳴𑌰𑌾 \ul{𑌹}\-𑌤𑌾𑌘᳴𑌶𑍞𑌸𑌾𑌵𑌾\-\ul{𑌭}\-𑌰𑌦𑍍𑌵᳴𑌸𑍂 𑌵\-\ul{𑌸𑍁}\-𑌵𑌨𑍇᳴ 𑌵\-\ul{𑌸𑍁}\-𑌧𑍇𑌯᳴𑌸𑍍𑌯 𑌵𑍀\-\ul{𑌤𑌾𑌂} 𑌯𑌜᳴।
\-\ul{𑌦𑍇}\-𑌵𑍀\-\ul{𑌸𑍍𑌤𑌿}\-𑌸𑍍𑌰\-\ul{𑌸𑍍𑌤𑌿}\-𑌸𑍍𑌰𑍋 \ul{𑌦𑍇}\-𑌵𑍀𑌰𑌿\-\ul{𑌡𑌾} 𑌸𑌰᳴𑌸𑍍𑌵\-\ul{𑌤𑍀} 𑌭𑌾𑌰᳴\-\ul{𑌤𑍀} 𑌦𑍍𑌯𑌾𑌂 𑌭𑌾𑌰᳴𑌤𑍍𑌯𑌾\-\ul{𑌦𑌿}\-𑌤𑍍𑌯𑍈𑌰᳴𑌸𑍍𑌪𑍃\-\ul{𑌕𑍍𑌷}\-𑌥𑍍𑌸𑌰᳴𑌸𑍍𑌵\-\ul{𑌤𑍀}\-𑌮𑍞 \ul{𑌰𑍁}\-𑌦𑍍𑌰𑍈\-\ul{𑌰𑍍𑌯}\-𑌜𑍍𑌞𑌮𑌾᳴𑌵𑍀\-\ul{𑌦𑌿}\-𑌹𑍈𑌵𑍇𑌡᳴\-\ul{𑌯𑌾} 𑌵𑌸𑍁᳴𑌮𑌤𑍍𑌯𑌾 𑌸\-\ul{𑌧}\-𑌮𑌾𑌦𑌂᳴ 𑌮𑌦𑍇𑌮 𑌵\-\ul{𑌸𑍁}\-𑌵𑌨𑍇᳴ 𑌵\-\ul{𑌸𑍁}\-𑌧𑍇𑌯᳴𑌸𑍍𑌯 𑌵𑌿𑌯\-\ul{𑌨𑍍𑌤𑍁} 𑌯𑌜᳴।
\-\ul{𑌦𑍇}\-𑌵𑍋 𑌨\-\ul{𑌰𑌾}\-𑌶𑍞𑌸᳴𑌸𑍍𑌤𑍍𑌰𑌿\-\ul{𑌶𑍀}\-\-\ul{𑌰𑍍}\-𑌷𑌾 𑌷᳴\-\ul{𑌡}\-𑌕𑍍𑌷𑌃 \ul{𑌶}\-𑌤𑌮𑌿𑌦𑍇᳴𑌨𑍞𑌶𑌿𑌤𑌿\-\ul{𑌪𑍃}\-𑌷𑍍𑌠𑌾 𑌆𑌦᳴𑌧𑌤𑌿 \ul{𑌸}\-𑌹𑌸𑍍𑌰᳴\-\ul{𑌮𑍀𑌂} 𑌪𑍍𑌰𑌵᳴𑌹𑌨𑍍𑌤𑌿 \ul{𑌮𑌿}\-𑌤𑍍𑌰𑌾𑌵\-\ul{𑌰𑍁}\-𑌣𑍇𑌦᳴𑌸𑍍𑌯 \ul{𑌹𑍋}\-𑌤𑍍𑌰𑌮𑌰𑍍\mbox{}𑌹᳴\-\ul{𑌤𑍋} 𑌬𑍃\-\ul{𑌹}\-𑌸𑍍𑌪𑌤𑌿𑌃᳴ \ul{𑌸𑍍𑌤𑍋}\-𑌤𑍍𑌰\-\ul{𑌮}\-𑌶𑍍𑌵𑌿𑌨𑌾\-𑌽𑌽𑌧𑍍𑌵᳴𑌰𑍍𑌯𑌵𑌂 𑌵\-\ul{𑌸𑍁}\-𑌵𑌨𑍇᳴ 𑌵\-\ul{𑌸𑍁}\-𑌧𑍇𑌯𑌸𑍍𑌯᳴ 𑌵𑍇\-\ul{𑌤𑍁} 𑌯𑌜᳴।
\-\ul{𑌦𑍇}\-𑌵𑍋 𑌵\-\ul{𑌨}\-𑌸𑍍𑌪𑌤𑌿᳴\-\ul{𑌰𑍍𑌵}\-\-\ul{𑌰𑍍}\-𑌷𑌪𑍍𑌰𑌾᳴𑌵𑌾 \ul{𑌘𑍃}\-𑌤𑌨𑌿᳴\-\ul{𑌰𑍍𑌣𑌿}\-𑌗𑍍𑌦𑍍𑌯𑌾𑌮\-\ul{𑌗𑍍𑌰𑍇}\-𑌣𑌾𑌸𑍍𑌪𑍃᳴\-\ul{𑌕𑍍𑌷}\-𑌦𑌾𑌨𑍍𑌤𑌰𑌿᳴\-\ul{𑌕𑍍𑌷𑌂} 𑌮𑌧𑍍𑌯𑍇᳴𑌨𑌾𑌪𑍍𑌰𑌾𑌃 𑌪𑍃\-\ul{𑌥𑌿}\-𑌵𑍀𑌮𑍁𑌪᳴𑌰𑍇𑌣𑌾𑌦𑍃𑍞𑌹𑍀𑌦𑍍𑌵\-\ul{𑌸𑍁}\-𑌵𑌨𑍇᳴ 𑌵\-\ul{𑌸𑍁}\-𑌧𑍇𑌯᳴𑌸𑍍𑌯 𑌵𑍇\-\ul{𑌤𑍁} 𑌯𑌜᳴।
\-\ul{𑌦𑍇}\-𑌵𑌂 \ul{𑌬}\-\-\ul{𑌰𑍍}\-𑌹𑌿𑌰𑍍𑌵𑌾𑌰𑌿᳴𑌤𑍀𑌨𑌾𑌂 \ul{𑌨𑌿}\-𑌧𑍇𑌧𑌾᳴𑌽\-\ul{𑌸𑌿} 𑌪𑍍𑌰𑌚𑍍𑌯𑍁᳴𑌤𑍀\-\ul{𑌨𑌾}\-𑌮𑌪𑍍𑌰᳴\-𑌚𑍍𑌯𑍁𑌤𑌨𑍍𑌨𑌿𑌕𑌾\-\ul{𑌮}\-𑌧𑌰᳴𑌣𑌂 𑌪𑍁𑌰𑍁\-\ul{𑌸𑍍𑌪𑌾}\-\-\ul{𑌰𑍍}\-𑌹𑌂 𑌯𑌶᳴𑌸𑍍𑌵\-\ul{𑌦𑍇}\-𑌨𑌾 \ul{𑌬}\-\-\ul{𑌰𑍍}\-𑌹𑌿\-\ul{𑌷𑌾}\-\-𑌽𑌨𑍍𑌯𑌾 \ul{𑌬}\-\-\ul{𑌰𑍍}\-𑌹𑍀𑍟\-\ul{𑌷𑍍𑌯}\-𑌭𑌿 𑌷𑍍𑌯𑌾᳴𑌮 𑌵\-\ul{𑌸𑍁}\-𑌵𑌨𑍇᳴ 𑌵\-\ul{𑌸𑍁}\-𑌧𑍇𑌯᳴𑌸𑍍𑌯 𑌵𑍇\-\ul{𑌤𑍁} 𑌯𑌜᳴।
\-\ul{𑌦𑍇}\-𑌵𑍋 \ul{𑌅}\-𑌗𑍍𑌨𑌿𑌃 𑌸𑍍𑌵𑌿᳴\-\ul{𑌷𑍍𑌟}\-𑌕𑍃\-\ul{𑌥𑍍𑌸𑍁}\-𑌦𑍍𑌰𑌵𑌿᳴𑌣𑌾 \ul{𑌮}\-𑌨𑍍𑌦𑍍𑌰𑌃 \ul{𑌕}\-𑌵𑌿𑌃 \ul{𑌸}\-𑌤𑍍𑌯𑌮᳴𑌨𑍍𑌮𑌾\-𑌽𑌽\-\ul{𑌯}\-𑌜𑍀 𑌹𑍋\-\ul{𑌤𑌾} 𑌹𑍋𑌤𑍁᳴\-\ul{𑌰𑍍}\-𑌹𑍋\-\ul{𑌤𑍁}\-𑌰𑌾𑌯᳴𑌜𑍀\-\ul{𑌯𑌾}\-𑌨\-\ul{𑌗𑍍𑌨𑍇} 𑌯𑌾\-\ul{𑌨𑍍𑌦𑍇}\-𑌵𑌾𑌨\-\ul{𑌯𑌾}\-𑌡𑍍𑌯𑌾𑍞 𑌅𑌪𑌿᳴\-\ul{𑌪𑍍𑌰𑍇}\-𑌰𑍍𑌯𑍇 𑌤𑍇᳴ \ul{𑌹𑍋}\-𑌤𑍍𑌰𑍇 𑌅𑌮᳴𑌥𑍍𑌸\-\ul{𑌤} 𑌤𑌾𑍞 𑌸᳴\-\ul{𑌸}\-𑌨𑍁\-\ul{𑌷𑍀}\-\-\ul{𑍞} 𑌹𑍋𑌤𑍍𑌰𑌾𑌂 𑌦𑍇𑌵\-\ul{𑌙𑍍𑌗}\-𑌮𑌾\-\ul{𑌨𑍍𑌦𑌿}\-𑌵𑌿 \ul{𑌦𑍇}\-𑌵𑍇𑌷𑍁᳴ \ul{𑌯}\-𑌜𑍍𑌞𑌮𑍇𑌰᳴\-\ul{𑌯𑍇}\-𑌮𑍟 𑌸𑍍𑌵𑌿᳴\-\ul{𑌷𑍍𑌟}\-𑌕𑍃𑌚𑍍𑌚𑌾\-\ul{𑌗𑍍𑌨𑍇} 𑌹𑍋𑌤𑌾\-𑌽𑌭𑍂॑𑌰𑍍𑌵\-\ul{𑌸𑍁}\-𑌵𑌨𑍇᳴ 𑌵\-\ul{𑌸𑍁}\-𑌧𑍇𑌯᳴𑌸𑍍𑌯 𑌨𑌮𑍋\-\ul{𑌵𑌾}\-𑌕𑍇 𑌵𑍀\-\ul{𑌹𑌿} 𑌯𑌜᳴॥33॥\anuvakamend[𑌯𑌜𑍈𑌕𑌂᳴ 𑌚]

%3.6.14.1
\-\ul{𑌦𑍇}\-𑌵𑌂 \ul{𑌬}\-\-\ul{𑌰𑍍}\-𑌹𑌿𑌃।
\-\ul{𑌵}\-\-\ul{𑌸𑍁}\-𑌵𑌨𑍇᳴ 𑌵\-\ul{𑌸𑍁}\-𑌧𑍇𑌯᳴𑌸𑍍𑌯 𑌵𑍇𑌤𑍁।
\-\ul{𑌦𑍇}\-𑌵𑍀𑌰𑍍𑌦𑍍𑌵𑌾𑌰𑌃᳴।
\-\ul{𑌵}\-\-\ul{𑌸𑍁}\-𑌵𑌨𑍇᳴ 𑌵\-\ul{𑌸𑍁}\-𑌧𑍇𑌯᳴𑌸𑍍𑌯 𑌵𑌿𑌯𑌨𑍍𑌤𑍁।
\-\ul{𑌦𑍇}\-𑌵𑍀 \ul{𑌉}\-𑌷𑌾\-\ul{𑌸𑌾}\-𑌨𑌕𑍍𑌤𑌾॑।
\-\ul{𑌵}\-\-\ul{𑌸𑍁}\-𑌵𑌨𑍇᳴ 𑌵\-\ul{𑌸𑍁}\-𑌧𑍇𑌯᳴𑌸𑍍𑌯 𑌵𑍀𑌤𑌾𑌮𑍍।
\-\ul{𑌦𑍇}\-𑌵𑍀 𑌜𑍋𑌷𑍍𑌟𑍍𑌰𑍀॑।
\-\ul{𑌵}\-\-\ul{𑌸𑍁}\-𑌵𑌨𑍇᳴ 𑌵\-\ul{𑌸𑍁}\-𑌧𑍇𑌯᳴𑌸𑍍𑌯 𑌵𑍀𑌤𑌾𑌮𑍍।
\-\ul{𑌦𑍇}\-𑌵𑍀 \ul{𑌊}\-𑌰𑍍𑌜𑌾𑌹𑍁᳴𑌤𑍀।
\-\ul{𑌵}\-\-\ul{𑌸𑍁}\-𑌵𑌨𑍇᳴ 𑌵\-\ul{𑌸𑍁}\-𑌧𑍇𑌯𑌸𑍍𑌯᳴ 𑌵𑍀𑌤𑌾𑌮𑍍॥34॥

%3.6.14.2
\-\ul{𑌦𑍇}\-𑌵𑌾 𑌦𑍈\-\ul{𑌵𑍍𑌯𑌾} 𑌹𑍋𑌤𑌾᳴𑌰𑌾।
\-\ul{𑌵}\-\-\ul{𑌸𑍁}\-𑌵𑌨𑍇᳴ 𑌵\-\ul{𑌸𑍁}\-𑌧𑍇𑌯᳴𑌸𑍍𑌯 𑌵𑍀𑌤𑌾𑌮𑍍।
\-\ul{𑌦𑍇}\-𑌵𑍀\-\ul{𑌸𑍍𑌤𑌿}\-𑌸𑍍𑌰\-\ul{𑌸𑍍𑌤𑌿}\-𑌸𑍍𑌰𑍋 \ul{𑌦𑍇}\-𑌵𑍀𑌃।
\-\ul{𑌵}\-\-\ul{𑌸𑍁}\-𑌵𑌨𑍇᳴ 𑌵\-\ul{𑌸𑍁}\-𑌧𑍇𑌯᳴𑌸𑍍𑌯 𑌵𑌿𑌯𑌨𑍍𑌤𑍁।
\-\ul{𑌦𑍇}\-𑌵𑍋 𑌨\-\ul{𑌰𑌾}\-𑌶𑍞𑌸𑌃᳴।
\-\ul{𑌵}\-\-\ul{𑌸𑍁}\-𑌵𑌨𑍇᳴ 𑌵\-\ul{𑌸𑍁}\-𑌧𑍇𑌯᳴𑌸𑍍𑌯 𑌵𑍇𑌤𑍁।
\-\ul{𑌦𑍇}\-𑌵𑍋 𑌵\-\ul{𑌨}\-𑌸𑍍𑌪𑌤𑌿𑌃᳴।
\-\ul{𑌵}\-\-\ul{𑌸𑍁}\-𑌵𑌨𑍇᳴ 𑌵\-\ul{𑌸𑍁}\-𑌧𑍇𑌯᳴𑌸𑍍𑌯 𑌵𑍇𑌤𑍁।
\-\ul{𑌦𑍇}\-𑌵𑌂 \ul{𑌬}\-\-\ul{𑌰𑍍}\-𑌹𑌿𑌰𑍍𑌵𑌾𑌰𑌿᳴𑌤𑍀𑌨𑌾𑌮𑍍।
\-\ul{𑌵}\-\-\ul{𑌸𑍁}\-𑌵𑌨𑍇᳴ 𑌵\-\ul{𑌸𑍁}\-𑌧𑍇𑌯᳴𑌸𑍍𑌯 𑌵𑍇𑌤𑍁॥35॥

%3.6.14.3
\-\ul{𑌦𑍇}\-𑌵𑍋 \ul{𑌅}\-𑌗𑍍𑌨𑌿𑌃 𑌸𑍍𑌵𑌿᳴\-\ul{𑌷𑍍𑌟}\-𑌕𑍃𑌤𑍍।
\-\ul{𑌸𑍁}\-𑌦𑍍𑌰𑌵𑌿᳴𑌣𑌾 \ul{𑌮}\-𑌨𑍍𑌦𑍍𑌰𑌃 \ul{𑌕}\-𑌵𑌿𑌃।
\-\ul{𑌸}\-𑌤𑍍𑌯𑌮᳴𑌨𑍍𑌮𑌾\-\ul{𑌯}\-𑌜𑍀 𑌹𑍋𑌤𑌾॑।
𑌹𑍋𑌤𑍁᳴𑌰𑍍‌\mbox{}𑌹𑍋\-\ul{𑌤𑍁}\-𑌰𑌾𑌯᳴𑌜𑍀𑌯𑌾𑌨𑍍।
𑌅\-\ul{𑌗𑍍𑌨𑍇} 𑌯𑌾\-\ul{𑌨𑍍𑌦𑍇}\-𑌵𑌾𑌨𑌯𑌾॑𑌟𑍍।
𑌯𑌾𑍞 𑌅𑌪𑌿᳴𑌪𑍍𑌰𑍇𑌃।
𑌯𑍇 𑌤𑍇᳴ \ul{𑌹𑍋}\-𑌤𑍍𑌰𑍇 𑌅𑌮᳴𑌥𑍍𑌸𑌤।
𑌤𑌾𑍞 𑌸᳴\-\ul{𑌸}\-𑌨𑍁\-\ul{𑌷𑍀}\-\-\ul{𑍞} 𑌹𑍋𑌤𑍍𑌰𑌾॑𑌨𑍍𑌦𑍇𑌵\-\ul{𑌙𑍍𑌗}\-𑌮𑌾𑌮𑍍।
\-\ul{𑌦𑌿}\-𑌵𑌿 \ul{𑌦𑍇}\-𑌵𑍇𑌷𑍁᳴ \ul{𑌯}\-𑌜𑍍𑌞𑌮𑍇𑌰᳴\-\ul{𑌯𑍇}\-𑌮𑌮𑍍।
\-\ul{𑌸𑍍𑌵𑌿}\-\-\ul{𑌷𑍍𑌟}\-𑌕𑍃𑌚𑍍𑌚𑌾\-\ul{𑌗𑍍𑌨𑍇} 𑌹𑍋𑌤𑌾\-𑌽𑌭𑍂𑌃॑।
\-\ul{𑌵}\-\-\ul{𑌸𑍁}\-𑌵𑌨𑍇᳴ 𑌵\-\ul{𑌸𑍁}\-𑌧𑍇𑌯᳴𑌸𑍍𑌯 𑌨𑌮𑍋\-\ul{𑌵𑌾}\-𑌕𑍇 𑌵𑍀𑌹𑌿᳴॥36॥\anuvakamend[\-\ul{𑌵𑍀}\-\-\ul{𑌤𑌾𑌂} \ul{𑌵𑍇}\-𑌤𑍍𑌵\-\ul{𑌭𑍂}\-𑌰𑍇𑌕𑌂᳴ 𑌚]

%3.6.15.1
\-\ul{𑌅}\-𑌗𑍍𑌨𑌿\-\ul{𑌮}\-𑌦𑍍𑌯 𑌹𑍋𑌤𑌾᳴𑌰𑌮𑌵𑍃𑌣𑍀\-\ul{𑌤𑌾}\-𑌯𑌂 𑌯𑌜᳴𑌮𑌾\-\ul{𑌨𑌃} 𑌪𑌚᳴\-\ul{𑌨𑍍𑌪}\-𑌕𑍍𑌤𑍀𑌃 𑌪𑌚᳴𑌨𑍍𑌪𑍁\-\ul{𑌰𑍋}\-𑌡𑌾𑌶𑌂᳴ \ul{𑌬}\-𑌧𑍍𑌨𑌨𑍍𑌨𑌿᳴\-\ul{𑌨𑍍𑌦𑍍𑌰𑌾}\-𑌗𑍍𑌨𑌿\-\ul{𑌭𑍍𑌯𑌾𑌂} 𑌛𑌾𑌗𑍞᳴ 𑌸𑍂\-\ul{𑌪}\-𑌸𑍍𑌥𑌾 \ul{𑌅}\-𑌦𑍍𑌯 \ul{𑌦𑍇}\-𑌵𑍋 𑌵\-\ul{𑌨}\-𑌸𑍍𑌪𑌤𑌿᳴𑌰𑌭𑌵𑌦𑌿\-\ul{𑌨𑍍𑌦𑍍𑌰𑌾}\-𑌗𑍍𑌨𑌿\-\ul{𑌭𑍍𑌯𑌾𑌂} 𑌛𑌾\-\ul{𑌗𑍇}\-𑌨𑌾𑌘᳴\-\ul{𑌸𑍍𑌤𑌾}\-𑌨𑍍𑌤𑌂 𑌮𑍇᳴\-\ul{𑌦}\-𑌸𑍍𑌤𑌃 𑌪𑍍𑌰𑌤𑌿᳴\-𑌪\-\ul{𑌚}\-𑌤𑌾𑌗𑍍𑌰᳴𑌭𑍀\-\ul{𑌷𑍍𑌟𑌾}\-𑌮𑌵𑍀᳴𑌵𑍃𑌧𑍇𑌤𑌾𑌂 𑌪𑍁\-\ul{𑌰𑍋}\-𑌡𑌾𑌶𑍇᳴\-\ul{𑌨} 𑌤𑍍𑌵𑌾\-\ul{𑌮}\-𑌦𑍍𑌯𑌰𑍍\mbox{}𑌷᳴ 𑌆𑌰𑍍\mbox{}𑌷𑍇𑌯 𑌋𑌷𑍀𑌣𑌾𑌨𑍍𑌨𑌪𑌾𑌦𑌵𑍃𑌣𑍀\-\ul{𑌤𑌾}\-𑌯𑌂 𑌯𑌜᳴𑌮𑌾𑌨𑍋 \ul{𑌬}\-𑌹𑍁\-\ul{𑌭𑍍𑌯} 𑌆 𑌸𑌙𑍍𑌗᳴𑌤𑍇𑌭𑍍𑌯 \ul{𑌏}\-𑌷 𑌮𑍇᳴ \ul{𑌦𑍇}\-𑌵𑍇\-\ul{𑌷𑍁} 𑌵\-\ul{𑌸𑍁} 𑌵𑌾𑌰𑍍𑌯𑌾 𑌯᳴𑌕𑍍𑌷𑍍𑌯\-\ul{𑌤} 𑌇\-\ul{𑌤𑌿} 𑌤𑌾 𑌯𑌾 \ul{𑌦𑍇}\-𑌵𑌾 𑌦𑍇᳴\-\ul{𑌵}\-𑌦𑌾\-\ul{𑌨𑌾}\-𑌨𑍍𑌯\-\ul{𑌦𑍁}\-𑌸𑍍𑌤𑌾𑌨𑍍𑌯᳴\-\ul{𑌸𑍍𑌮𑌾} 𑌆 \ul{𑌚} 𑌶𑌾𑌸𑍍𑌵𑌾 𑌚᳴ 𑌗𑍁𑌰𑌸𑍍𑌵𑍇\-\ul{𑌷𑌿}\-𑌤𑌶𑍍𑌚᳴ 𑌹𑍋\-\ul{𑌤}\-𑌰𑌸𑌿᳴ 𑌭\-\ul{𑌦𑍍𑌰}\-𑌵𑌾𑌚𑍍𑌯𑌾᳴\-\ul{𑌯} 𑌪𑍍𑌰𑍇𑌷𑌿᳴\-\ul{𑌤𑍋} 𑌮𑌾𑌨𑍁᳴𑌷𑌃 𑌸𑍂𑌕𑍍𑌤\-\ul{𑌵𑌾}\-𑌕𑌾𑌯᳴ \ul{𑌸𑍂}\-𑌕𑍍𑌤𑌾 𑌬𑍍𑌰𑍂᳴𑌹𑌿॥37॥\anuvakamend[\-\ul{𑌅}\-𑌗𑍍𑌨𑌿\-\ul{𑌮}\-𑌦𑍍𑌯𑍈𑌕𑌮𑍍॑]






\prashnaend{\-\ul{𑌅}\-𑌞𑍍𑌜\-\ul{𑌨𑍍𑌤𑌿} 𑌹𑍋𑌤𑌾᳴ 𑌯\-\ul{𑌕𑍍𑌷}\-𑌥𑍍𑌸𑌮𑌿᳴𑌦𑍍𑌧𑍋 \ul{𑌅}\-𑌦𑍍𑌯𑌾𑌗𑍍𑌨𑌿𑌰\-\ul{𑌜𑍈}\-𑌦𑍍𑌦𑍈𑌵𑍍𑌯𑌾᳴ \ul{𑌜𑍁}\-𑌷𑌸𑍍𑌵𑌾 𑌵𑍃᳴𑌤𑍍𑌰𑌹𑌣𑌾 \ul{𑌗𑍀}\-𑌰𑍍𑌭𑌿𑌸𑍍𑌤𑍍𑌵𑍟 𑌹𑍍𑌯𑌾𑌭᳴𑌰\-\ul{𑌤}\-𑌮𑍁𑌪𑍋᳴\-\ul{𑌹} 𑌯\-\ul{𑌦𑍍𑌦𑍇}\-𑌵𑌂 \ul{𑌬}\-\-\ul{𑌰𑍍}\-𑌹𑌿𑌃 𑌸𑍁᳴\-\ul{𑌦𑍇}\-𑌵𑌂 \ul{𑌦𑍇}\-𑌵𑌂 \ul{𑌬}\-\-\ul{𑌰𑍍}\-𑌹𑌿\-\ul{𑌰}\-𑌗𑍍𑌨𑌿\-\ul{𑌮}\-𑌦𑍍𑌯 𑌪𑌞𑍍𑌚᳴𑌦𑌶॥15॥}{\-\ul{𑌅}\-𑌞𑍍𑌜\-\ul{𑌨𑍍𑌤𑍍𑌯}\-𑌗𑍍𑌨𑌿𑌰𑍍\mbox{}𑌹𑍋𑌤𑌾᳴ 𑌨𑍋 \ul{𑌗𑍀}\-𑌰𑍍𑌭𑌿𑌰𑍁𑌪𑍋᳴ \ul{𑌹} 𑌯\-\ul{𑌦𑍍𑌵𑌿}\-𑌦𑌥𑌂᳴ \ul{𑌵𑌾}\-𑌜𑌿𑌨𑌃᳴ \ul{𑌸}\-𑌪𑍍𑌤𑌤𑍍𑌰𑌿𑍞᳴𑌶𑌤𑍍॥37॥}{\-\ul{𑌅}\-𑌞𑍍𑌜𑌨𑍍𑌤𑌿᳴ \ul{𑌸𑍂}\-𑌕𑍍𑌤𑌾𑌬𑍍𑌰𑍂᳴𑌹𑌿॥}{𑌹𑌰𑌿𑌃᳴ 𑌓𑌮𑍍॥}{𑌇𑌤𑌿 𑌶𑍍𑌰𑍀𑌕𑍃𑌷𑍍𑌣𑌯𑌜𑍁𑌰𑍍𑌵𑍇𑌦𑍀𑌯𑌤𑍈𑌤𑍍𑌤𑌿𑌰𑍀𑌯𑌬𑍍𑌰𑌾𑌹𑍍𑌮𑌣𑍇 𑌤𑍃𑌤𑍀𑌯𑌾𑌷𑍍𑌟𑌕𑍇 𑌷𑌷𑍍𑌠𑌃 𑌪𑍍𑌰𑌪𑌾𑌠𑌕𑌃 𑌸𑌮𑌾𑌪𑍍𑌤𑌃॥}
